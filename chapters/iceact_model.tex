% !TeX spellcheck = en_US
\chapter{The IceAct Model in \geant}

\section{\geant}
\geant is a multi-purpose simulation framework for the passage of particles trough matter, written in \textit{C++} and developed by the \geant Collaboration at CERN. It includes physics models, geometry, tracking, hits, and digitization and thus allows a detailed simulation and response analysis for particle detectors in many application fields like particle and accelerator physics, space engineering or medical science. In the framework's source some basic and advanced use cases are implemented and provided as examples. The toolkit is built up of multiple categories (or modules) using each other (cf. figure \ref{geant4:categories}). \cite{geant4}

\begin{wrapfigure}{r}{0.5\textwidth}
	\centering
	\includegraphics[width=0.5\textwidth]{Geant4Logo.png}
	\caption[\geant Logo]{\textbf{\geant logo.} \cite{geant4:logo}}	
\end{wrapfigure}

\begin{figure}[h]
	\centering
	\includegraphics[width=0.6\textwidth]{Geant4ConceptDiagram.pdf}
	\caption[\geant category diagram]{\textbf{Diagram of relationships between \geant categories. \cite{geant4}} The circles represent a \enquote{using} relation. The category with the circle next to the box uses the linked one.}	
	\label{geant4:categories}
\end{figure}

Especially for the use case of IceAct \geant is capable of simulating Cherenkov (optical) photons, material properties like transmission, reflection, and refraction, as well as detection efficiency properties of the SiPMs.

Since this thesis is about an approach of an all-encompassing telescope simulation, \geant provides all major possibilities to get a distinct analysis of the entire optical system of IceAct.

\subsection{FAMOUS telescope simulation}
The fluorescence telescope FAMOUS\footnote{\textbf{F}irst \textbf{A}uger \textbf{M}ulti-pixel
photon counter camera for the \textbf{O}bservation of \textbf{U}ltra-high-energy air \textbf{S}howers} for the Pierre Auger Observatory in Argentina is developed at RWTH Aachen to measure fluorescence light originating from ultra-high-energy cosmic rays (UHECR) by using Silicon Photomultipliers (SiPMs). Withing the development, a detailed \geant simulation has been elaborated. \cite{famous:sim_github,famous:sim_github} The telescope design of FAMOUS is very similar since the detection technique and the optics system is basically the same. Therefore, the IceAct telescope simulation is heavily based on this FAMOUS \geant framework. A detailed discourse and a summary of previous analyses can be found in \cite{famous:niggemann}.

\section{Materials}

For an optical device, the material that the light should pass has to be chosen deliberately. Especially the transmission properties, processability, and for IceAct in particular the resistance against harsh weather conditions are of interest.

The glass plate on top of IceAct is made of SCHOTT BOROFLOAT\textsuperscript{\textregistered} 33 borosilicate glass with a thickness of \SI{2+-0.2}{\milli\meter} and a diameter of \SI{650.3+-1}{\milli\meter}. Borosilicate is chosen for its high durability, transparency in the interesting spectral region, flatness, and weak fluorescence intensities. The refractive index is evaluated at some wavelength. Since we need to have a full dispersion relation the points are spline interpolated (cf. orange curve in figure \ref{iceact:model:material:refractive_index}). \cite{iceact:borosilicate:datasheet}. 

In the data sheet \cite{iceact:borosilicate:datasheet} the transmission properties are given for a vertical light and a glass plate of a thickness of $d = \SI{6.5}{\milli\meter}$. Therefore, the transmission curve $T_\text{total}(\lambda)$ includes the internal absorption as well as the two interface transitions into and out of the borosilicate.
\begin{align}
	T_\text{total}(\lambda) = T_\text{interface}^2(\lambda)\cdot T_\text{internal}(d=\SI{6.5}{\milli\meter},\lambda)
	\label{eq:transmission}
\end{align}
The transmission at the interface can be calculated by using the Fresnel equations. In case of perpendicular light, it is
\begin{align}
	T_\text{interface}(\lambda) = 1 - \left(\frac{n(\lambda)-n_\text{air}}{n(\lambda)+n_\text{air}}\right)^2
	\label{eq:perp_interface_transmission}
\end{align}
In \geant the wavelength dependent absorption length $a(\lambda)$ has to be implemented which is given by the exponential absorption law
\begin{align}
	I(x) = I_0 e^{-\frac{x}{a}} \Leftrightarrow a = - \frac{x}{\ln\left(\frac{I(x)}{I_0}\right)}
	\label{eq:absorptionlaw}
\end{align}
Thus, one gets the absorption length by using equations \eqref{eq:transmission}, \eqref{eq:perp_interface_transmission}, and \eqref{eq:absorptionlaw}.
\begin{align}
	a(\lambda) &= - \frac{d}{\ln T_\text{internal}(d,\lambda)}\\
	&= -\frac{d}{\ln T_\text{total}(\lambda)-2\ln\left(1 - \left(\frac{n(\lambda)-n_\text{air}}{n(\lambda)+n_\text{air}}\right)^2\right)}\,,
\end{align}
which is implemented in \geant material properties with $n_\text{air} = 1$. Figure \ref{iceact:model:material:transmission} shows the three transmission components as orange lines.

The Fresnel lens and the Winston Cones in IceAct are made of polymethyl methacrylate (PMMA, acrylic, or plexiglass). The dispersion $n(\lambda)$ can be parametrized with the empirical \textit{Sellmeier equation}. For glasses the usual form is
\begin{align}
	n^2(\lambda) = 1 + \frac{B_1\lambda^2}{\lambda^2-C_1} + \frac{B_2\lambda^2}{\lambda^2-C_2} + \frac{B_3\lambda^2}{\lambda^2-C_3}\,,
	\label{eq:sellmeier}
\end{align}
with the \textit{Sellmeier coefficients} $B_{1,2,3}$ and $C_{1,2,3}$. \cite{iceact:sellmeier} Table \ref{iceact:model:pmma_sellmeiercoeffs} shows the used coefficients and the function is plotted in figure \ref{iceact:model:material:refractive_index} (blue curve).

\begin{table}[h]
	\centering
	\begin{tabular}{c|l}
		$B_1$  & \num{0.99654}  \\
		$B_2$  & \num{0.18964}  \\
		$B_3$  & \num{0.00411}  \\
		$C_1$  & \SI{0.00787}{\micro\meter\squared}  \\
		$C_2$  & \SI{0.02191}{\micro\meter\squared}  \\
		$C_3$  & \SI{3.85727}{\micro\meter\squared}  \\
	\end{tabular}
	\caption[Sellmeier coefficients for PMMA]{\textbf{Sellmeier coefficients for PMMA.} \cite{iceact:refractiveindex} The above-mentioned coefficients are used in the \geant material properties for PMMA. The related Sellmeier equation \eqref{eq:sellmeier} is plotted in figure \ref{iceact:model:material:refractive_index} as the blue curve.}
	\label{iceact:model:pmma_sellmeiercoeffs}
\end{table}

For the transmission properties of PMMA, the same method as for borosilicate is used (see above). Therefore, the data stated in \cite{famous:niggemann} is taken as $T_\text{internal}(d = \SI{3}{\milli\meter})$. Figure \ref{iceact:model:material:transmission} shows the three transmission components as blue lines.

\begin{figure}[h]
	\centering
	\includegraphics[width=0.7\textwidth]{material/transmission.pdf}
	\caption[Transmission of used materials]{\textbf{Transmission functions of materials used in the simulation.} The total transmission function is the product of internal and two interface transmissions which is evaluated for a perpendicularly incident particle in this plot. Thus, the solid lines represent a complete (perpendicular) transition through a \SI{3}{\milli\meter} thick layer of the respective material. For a better comparison, the data of internal transmission for borosilicate is converted from \SI{6.5}{\milli\meter} into \SI{3}{\milli\meter}.}
	\label{iceact:model:material:transmission}	
\end{figure}

\begin{figure}[h]
	\centering
	\includegraphics[width=0.7\textwidth]{material/refractive_index.pdf}
	\caption[Refractive index of used materials]{\textbf{Refractive index of materials used in the simulation.} For PMMA the dispersion is calculated by evaluating the Sellmeier equation introduced in this section. The refractive index for the used borosilicate is only given at specific wavelengths. \cite{iceact:borosilicate:datasheet} Therefore, spline interpolation is used to reconstruct the full curve.}
	\label{iceact:model:material:refractive_index}	
\end{figure}

The tube, back plane and other coating surfaces are simulated as \enquote{dummy} material with no reflection or transmission parameters. A particle that hits those surfaces is absorbed and not considered any further.

\section{Optics}

In the following section the four optical components of the IceAct \geant model are discussed. To have a first glimpse of the model, see figure \ref{iceact:model:cut}.
\begin{figure}[h]
	\centering
	\includegraphics[width=0.6\textwidth]{IceActGeant4Model.pdf}
	\caption[IceAct \geant model]{\textbf{The IceAct \geant model.} Cross-sectional sketch of the IceAct optics in \geant with all simulated components. They are described in detail in \cref{iceact:model:fresnellens,iceact:model:camera,iceact:model:glassplate}.}
	\label{iceact:model:cut}	
\end{figure}

\subsection{Glass Plate}\label{iceact:model:glassplate}

\todo{glass plate}

\subsection{Fresnel Lens}\label{iceact:model:fresnellens}

\begin{figure}[h]
	\centering
	\includegraphics[width=0.5\textwidth]{FresnelVsNormalLens.pdf}
	\caption[Comparison conventional vs. Fresnel lens]{\textbf{Comparison between a conventional \enquote{thick} lens and a Fresnel lens. \cite{famous:eichler}} For the functionality of a lens the radius-dependent sagitta function $z(\rho)$ is crucial. To get rid of the bulky material of a conventional lens, the Fresnel lens is divided into annular \enquote{prisms} called \enquote{grooves}. The slope angle $\delta$ of each groove is a local approximation of the sagitta function to ensure the imaging capability.}
	\label{iceact:model:fresnelvsthick}	
\end{figure}

\begin{figure}[h]
	\centering
	\includegraphics[width=0.5\textwidth]{FresnelGroove.pdf}
	\caption[Fresnel groove]{\textbf{Cross-sectional sketch of a Fresnel groove. \cite{famous:eichler}}}
	\label{iceact:model:fresnelgroove}	
\end{figure}

\todo{write}
\cite{famous:eichler}

\subsection{Camera}\label{iceact:model:camera}

\todo{write}

\subsubsection{Winston Cones}

\todo{write}

\subsubsection{Silicon Photomultiplier}

\todo{write}

