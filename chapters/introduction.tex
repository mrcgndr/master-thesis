% !TeX spellcheck = en_US
\chapter{Introduction}

\todo{maybe prologue?}

\section{Motivation}

\todo{Motivation}

\section{The \icecube Neutrino Observatory}

Since January 2011, the \icecube Neutrino Observatory at the Amundsen-Scott South Pole station is measuring neutrinos emanating from various sources. For this purpose a detector instrumented with digital optical modules (\enquote{DOMs}) is installed deep in the antarctic ice. 5160 of these optical sensors are arranged on 86 strings at a height between \SI{1450}{\meter} and \SI{2450}{\meter} below the surface. The central region of this In-Ice Array which has a higher density of DOMs is called \enquote{DeepCore}. Figure \ref{icecube:detector} shows a sketch of the detector arrangement.\\

\begin{figure}[h]
	\includegraphics[width=\textwidth]{IceCubeDetector.pdf}
	\caption[Schematic view of \icecube]{\textbf{Schematic view of the \icecube Neutrino Observatory.} \cite{icecube:instrumentation} The in-ice array with the denser sub-array DeepCore as well as the surface array \icetop is sketched. Different station colors represent different deployment stages.}
	\label{icecube:detector}
\end{figure}

Neutrinos are very interesting elementary particles because of their weak interaction cross section and their electrical neutrality. This fact makes it possible for neutrinos to let them point back to their sources which is exploited in the search for astrophysical processes like active galactic nuclei, supernovae, or gamma-ray bursts. Since they are able to reach us without scattering processes, neutrinos can even give information about sources at cosmological distances. Simultaneously, the weak interaction potential is what neutrino detection makes challenging. Therefore, a detector with a large scale active volume is needed. In the case of \icecube, this is \SI{1}{\cubic\kilo\meter} of ice.\\

At the surface on top of the in-ice detector the cosmic-ray air shower array \icetop is installed to detect Cherenkov radiation (cf. \ref{sec:cherenkov}). \icetop consists of 81 stations approximately arranged in the same grid as the in-ice strings. Each station has two tanks filled up with ice and two standard \icecube DOMs. This arrangement makes it possible for \icetop to detect primary cosmic rays (cf. \ref{sec:cosmicrays}) in the energy range of \si{\peta\electronvolt} to \si{\exa\electronvolt}. One purpose of \icetop is to provide a veto for downward-going neutrinos in the \icecube detector originating from coincident atmospheric air shower events. \cite{icecube:instrumentation} Since \icecube is mainly investigating astrophysical neutrinos, atmospheric neutrinos are a major background.\\

Since start of operation, \icecube has achieved two important breakthroughs. In 2013, high-energy extraterrestrial neutrinos were observed~\cite{icecube:he_neutrino}. Four years later -- on \num{22} September 2017 -- \icecube has detected a high-energy neutrino from the blazar \textit{TXS+0506-056} which was coincident in time and direction with a gamma-ray flare of this source. As a result, this was the first time that an extra-galactic accelerator could be identified as a source of an astrophysical neutrino~\cite{icecube:txs}.

\subsection{\icecube-Gen2}

\textit{\icecube-Gen2} is proposed to be a substantial enhancement of the \icecube Neutrino Observatory. It es planned to have a much better sensitivity to high-energy neutrinos by an instrumented volume about ten times bigger than \icecube has today. Studies of extra-galactic neutrino sources and resolving their location require a much higher rate of detected neutrino than \icecube provides by now. For this purpose, more strings with a denser spacing of further improved DOMs will be deployed covering a much lager in-ice volume.~\cite{icecube:iceact,icecube:gen2} Additionally, a large upgrade of \icetop concerning veto capabilities (\textit{IceVeto}) is discussed with different approaches. These are scintillation detectors (\textit{IceScint})~\cite{icecube:gen2:icescint}, a radio antenna array~\cite{icecube:gen2:radio}, or imaging air-Cherenkov telescopes (\textit{IceAct})~\cite{iacts:extension,icecube:iceact}.

\section{Cosmic Rays}\label{sec:cosmicrays}

Charged particles or nuclei that are propagating through the universe and incidentally reach the Earth's atmosphere are called \textit{cosmic rays}. They were discovered by the Austrian physicist \textsc{Victor Franz Hess} in 1912 when he observed an increasing discharge of electroscopes with increasing height in seven balloon flights. \cite{cosmicrays:hess} Hess initially called this underlying radiation \enquote{durchdringende Strahlung} (\textit{penetrating radiation}).

\begin{figure}[h]
	\includegraphics[width=\textwidth]{CosmicRayEnergySpectrum.pdf}
	\caption[Cosmic ray energy spectrum]{\textbf{Energy spectrum of cosmic rays measured with multiple air shower experiments.} \cite[adapted]{cosmicrays:gaisser} The three prominent regions known as \enquote{knee}, \enquote{second knee}, and \enquote{ankle} are marked. Multiplication of the spectrum by the factor $E^{2.6}$ leads to a better visibility and shows that the spectral index changes at these features.}
	\label{cosmicrays:spectrum}	
\end{figure}

When it comes to cosmic rays, ascertaining the mass composition is a key measurement for learning about their propagation in universe and about extra-galactic cosmic ray accelerators. Figure~\ref{cosmicrays:spectrum} shows that the energy spectrum of cosmic rays follows a power law:
\begin{align}
\frac{dE}{dN}\propto E^\gamma\,,
\end{align}
introducing a spectral index $\gamma$ which is dependent from the considered energy region. Measurements show \cite{cosmicrays:hoerandel, cosmicrays:fowler}
\begin{align}
	\gamma(E)=
	\begin{cases}
		\num{2.7} & \sim\SI{10}{\giga\electronvolt} < E < \SI{4e6}{\giga\electronvolt}\\
		\num{3.1} & \SI{4e6}{\giga\electronvolt} < E < \SI{4e9}{\giga\electronvolt}\\
		\num{2.6} & \SI{4e9}{\giga\electronvolt} < E < \SI{3e11}{\giga\electronvolt}
	\end{cases}\,.
\end{align}
Due to this interesting features, composition measurements at these \enquote{transition points} are desired in particular.

Due to the shape, the phenomenological model to describe the spectrum is referred to as \textit{poly gonato} (Greek for \enquote{many knees}). The \enquote{knee} is assumed to be based on different rigidity\footnote{property of a magnetic field to bend a particle's trajectory} dependent cut-off energies for sub-spectra of element groups which sum up to the spectrum we observe. \cite{cosmicrays:hoerandel, cosmicrays:shapiro} At energies beyond \SI{e11}{\giga\electronvolt} a strong suppression is observed. The GZK-effect (named after Kenneth \textsc{Greisen}, Georgiy \textsc{Zatsepin}, and Vadim \textsc{Kuzmin}) is supposed to be the reason. Protons with energies above a threshold of \SI{5e19}{\electronvolt} can interact with photons of the cosmic microwave background in such a way that they produce $\pi^0$ and $\pi^+$ mesons via $\Delta^+$ resonance:
\begin{subequations}
	\begin{align}
	\gamma_\text{CMB} + p \rightarrow \Delta^+ &\rightarrow p + \pi^0\\
	&\rightarrow n + \pi^+\,.
	\end{align}
\end{subequations}
Thus, the protons effectively lose about \SI{20}{\percent} of their energy. Additionally, calculations show that these interactions become quite frequent for proton energies of $E_p\gtrsim\SI{e20}{\electronvolt}$ which results in an effective cutoff of cosmic-ray energies above this region. \cite{cosmicrays:gzk}

\section{Extensive Air Showers}

If a high energetic particle -- a photon or hadron -- incidentally reaches the Earth's atmosphere, it interacts with their atoms. A common way to describe the traversed atmospheric matter for an air shower is the \textit{slant depth}
\begin{align}
X(h) = \int_{h}^{\infty}\rho(h')dh'\,,
\end{align}
with the height dependent air density $\rho(h)$. Once a high energetic \enquote{primary} particle interacts with an atmospheric atom, it initiates a cascade of secondary particles. Typically, one differentiates between hadronic and electro-magnetic cascades or showers (cf. figure \ref{airshowers:cascades}). For electro-magnetic showers or sub-showers \textit{Heitler's model} is used as a simple conception. The model is based on two-body splittings of electrons, positrons, and photons by $e^+ e^-$ pair production or bremsstrahlung which occur after a fixed distance $d=\lambda_\text{em}\ln{2}$ by using the medium-specific \textit{radiation length} $\lambda_\text{em}$. In other words: after $n$ splitting processes the shower consists out of $N = 2^n = e^{x/\lambda_\text{em}}$ electrons and photons.

\begin{figure}[h]
	\includegraphics[width=\textwidth]{AirShowerHeitler.pdf}
	\caption[Schematic view of extensive air showers]{\textbf{Schematic view of extensive air showers.} \cite{famous:niggemann} Two possible shower formations are shown. A primary photon initiates a electro-magnetic shower (left) whereas a primary proton initiates a hadronic shower with electro-magnetic sub-showers. The splitting steps are stated as well as the interaction lengths. For the hadronic cascades not all traces are sketched for clarity reasons.}	
	\label{airshowers:cascades}
\end{figure}

The multiplication process holds, until the particle energies are high enough for pair production and bremsstrahlung. Below this energy, which Heitler named as the \textit{critical energy}~$\xi_c^e$, the shower size decreases. Hence, the maximum number of particles $N_\text{max} = 2^{n_c}$ is reached after $n_c$ splitting steps. The energy of a considered primary photon $E_\circ$ is then distributed over all secondary shower particles so that $E_\circ = \xi_c^e N_\text{max} = \xi_c^e 2^{n_c}$. With this information, one can derive the slant depth $X_\text{max}$ at which the shower has the largest size. It is
\begin{align}
	X_\text{max}^\gamma = n_c\lambda_\text{em}\ln{2} = \lambda_\text{em}\ln{\frac{E_\circ}{\xi_c^e}}\,.
	\label{airshowers:eq:xmax}
\end{align}
It should be mentioned that this calculation only holds for pure electro-magnetic showers which is the reason for the superscript $\gamma$ in equation \eqref{airshowers:eq:xmax}. \cite{airshowers:heitlermodel}
Since the maximum depth $X_\text{max}$ is dependent from energy and type of the primary particle it is very important for composition studies of cosmic rays. A detailed discussion on several interaction models and comparisons to simulation is done in \cite{airshowers:heitlermodel}.

\section{Detection of Cosmic-ray Air Showers via Atmospheric Cherenkov Light}\label{sec:cherenkov}

\subsection{Detection Techniques}

Various detection techniques have been developed in the last decades. Overall, they can be divided into two categories:
\begin{itemize}
	\item Direct detection of air shower particles on ground or underground
	\item Detection of electromagnetic radiation originating directly or indirectly from the electromagnetic air shower component (and the muonic component in a minor part)
\end{itemize}
Commonly used particle detection experiments are based on scintillation or water-Cherenkov light.
For the second category, there are some different effects to look at, in particular fluorescence light, radio, and air-Cherenkov light (cf. section~\ref{sec:intro:cherenkov}) emission. Figure~\ref{intro:airshower_detection_sketch} gives a schematic overview of the operating techniques.~\cite{airshowers:schroeder}\\

\begin{figure}[h]
	\includegraphics[width=\textwidth]{AirShowerDetection.pdf}
	\caption[Different techniques for air shower detection]{\textbf{Different techniques for the detection of atmospheric air showers.} \cite{airshowers:schroeder} Most of the detection techniques (radio antennas, air-Cherenkov, and fluorescence light detectors) use the electromagnetic (e.m.) shower component. Muons typically reach further than the e.m. component and can be measured by particle detectors. There are more different particles than depicted contributing to real air showers. In addition, the right plot shows the longitudinal shower profile for hadrons, electrons/positrons, and muons/anti-muons.}	
	\label{intro:airshower_detection_sketch}
\end{figure}

Important parameters to be detected are direction and energy of air showers. Directional information can be reconstructed by measuring arrival times or an imaging system. The energy can be determined by the integrated signal strength, i.e. the amount of detected particles in the case of (under-)ground particle detectors or, respectively, the amount of electromagnetic radiation. Composition dependent air shower variables like $X_\text{max}$ are more difficult to detect.

A major challenge all detection varieties have in common is the low flux of primary cosmic rays. One tries to compensate for this by maximizing the instrumented or observed volume. Fluorescence light detectors typically have a large observable atmospheric volume and a sensitivity for air showers up to a few \SI{10}{\kilo\meter}. For other techniques, large surface arrays instrumenting areas up to several \SI{1000}{\cubic\kilo\meter} have to be build.~\cite{airshowers:schroeder}\\

In this thesis, the focus is on the air shower detection by Cherenkov light typically at optical and ultraviolet wavelengths. Thus, the underlying \textit{Cherenkov effect} is introduced in the next section.

\subsection{The Cherenkov Effect}\label{sec:intro:cherenkov}

The \textit{Cherenkov effect} is named after the Soviet physicist \textsc{Pavel A. Cherenkov} and describes the emission of radiation if a charged particle traverses a medium with a speed exceeding speed of light in that medium. \cite{airshowers:cherenkov} This is possible due to the fact that speed of light in a medium with a refractive index $n > 1$ is always below vacuum speed of light $c_0$ since
\begin{align}
	c = \frac{c_0}{n} \overset{n>1}{\Rightarrow} c < c_0\,.
\end{align} 
The effect is describable in two ways which are shown in figure \ref{airshowers:cherenkov}.
\begin{figure}[H]
	\centering
	\begin{subfigure}[t]{0.45\textwidth}
		\includegraphics[width=\textwidth]{Cherenkov1Pol.pdf}
		\subcaption{$v < \frac{c_0}{n}$ -- The charged particle traverses the dielectric medium and polarizes its environment. For the repolarization an electric field propagates with speed of light in the medium. Since the particle's velocity is slower, no net polarization is induced and therefore no radiation is emitted.}
		\label{airshowers:cherenkov1pol}
	\end{subfigure}
	\hfill
	\begin{subfigure}[t]{0.45\textwidth}
		\includegraphics[width=\textwidth]{Cherenkov2Pol.pdf}
		\subcaption{$v > \frac{c_0}{n}$ -- The charged particles traverses the dielectric medium and polarized its environment. Since its velocity is exceeding speed of light in the medium, the electric field repolarizes the surrounding particles not fast enough so that a net polarization is induced. Cherenkov photons are emitted under a fixed angle with respect to the trajectory of the charged particle.}
		\label{airshowers:cherenkov2pol}
	\end{subfigure}
	\vfill
	\begin{subfigure}[t]{0.45\textwidth}
		\includegraphics[width=\textwidth]{Cherenkov1Huy.pdf}
		\subcaption{$v < \frac{c_0}{n}$ -- The charged particle induces electro-magnetic elementary waves along its trajectory which propagate faster trough the medium than the particle. No radiation is emitted.}
		\label{airshowers:cherenkov1huy}
	\end{subfigure}
	\hfill
	\begin{subfigure}[t]{0.45\textwidth}
		\includegraphics[width=\textwidth]{Cherenkov2Huy.pdf}
		\subcaption{$v > \frac{c_0}{n}$ -- The charged particle induces electro-magnetic elementary waves along its trajectory which propagate slower trough the medium than the particle. All elementary waves add up to a wavefront under a fixed angle $\theta_C$. With this model the Cherenkov effect can be interpreted as the optical analogue for the \textit{sonic boom}.}
		\label{airshowers:cherenkov2huy}
	\end{subfigure}
	\caption[Illustration for the Cherenkov effect]{\textbf{Illustration for the Cherenkov effect.} A charged particle is traversing the medium with a refractive index $n$ from left to right with velocity $v$. $c_0$ is the vacuum speed of light. In (\subref{airshowers:cherenkov1pol}) and (\subref{airshowers:cherenkov2pol}) the dipole interpretation of the Cherenkov effect is shown. (\subref{airshowers:cherenkov1huy}) and (\subref{airshowers:cherenkov2huy}) show the effect by exploiting Huygens' principle.}
	\label{airshowers:cherenkov}
\end{figure}
A \textit{Cherenkov angle} $\theta_C$ -- as introduced in figure~\ref{airshowers:cherenkov2huy} -- can be calculated by applying trigonometry:
\begin{align}
	\cos{\theta_C} = \frac{c_0}{nv} = \frac{1}{n\beta}\,,
\end{align}
with the dimensionless velocity $\beta = \frac{v}{c_0}$.

Furthermore, the two Soviet physicists \textsc{Ilya M. Frank} and \textsc{Igor Y. Tamm} found a relation for the differential emission per wavelength and spatial interval known as the \textit{Frank-Tamm formula}~\cite{airshowers:franktamm}:
\begin{align}
	\frac{d^2N}{dxd\lambda} = 2\pi\alpha q^2 \frac{1}{\lambda^2}\left(1-\frac{1}{n^2(\lambda)\beta^2}\right)\,,
\end{align}
with
\begin{vardescription}
	\frac{d^2N}{dxd\lambda} & number of emitted Cherenkov photons per unit wavelength and unit propagation length,\\
	\alpha & fine structure constant,\\
	q & particle charge,\\
	\lambda & wavelength,\\
	n(\lambda) & refractive index of the medium (wavelength dependent),\\
	\beta=\frac{v}{c_0} & dimensionless relative velocity.\\
\end{vardescription}
The factor $\frac{1}{\lambda^2}$ suppresses higher wavelengths so that the Cherenkov spectrum is dominant in the ultra-violet regime. Figure \ref{airshowers:cherenkovspectrum} shows a measured energy spectrum.
\begin{figure}[h]
	\centering
	\includegraphics[width=0.7\textwidth]{CherenkovSpectrum.pdf}
	\caption[Cherenkov spectrum]{\textbf{Cherenkov wavelength spectrum.} Exemplary spectrum of Cherenkov photons measured at \SI{2200}{\meter} above sea level at the HEGRA IACT System (La Palma)\footnotemark. The falling edge towards high wavelength is proportional to $\lambda^{-2}$ whereas the falling edge towards low wavelengths is caused by atmospheric attenuation. Data adapted from \cite{airshowers:doering}.}	
	\label{airshowers:cherenkovspectrum}
\end{figure}
\footnotetext{\textbf{H}igh \textbf{E}nergy \textbf{G}amma \textbf{R}ay \textbf{A}stronomy, operated between 1987 and 2006 at Roque de los Muchachos Observatory on La Palma. \cite{airshowers:doering}}

The cone-like radiation profile of Cherenkov light with respect to the air shower axis makes it possible to reconstruct the shower direction by observing the direction of the Cherenkov photons.

\section{Imaging Air-Cherenkov Telescopes (IACTs)}

For the observation of cosmic-ray air showers and especially gamma-ray air showers, \textit{imaging air-Cherenkov telescopes} (\textit{IACT}s) are very suitable instruments. The common principle is to mount a camera looking down to a segmented mirror in order to detect reflecting Cherenkov photons with the ability to reconstruct their directions. State-of-the-art IACTs have mirrors with diameters up to \SI{17}{\meter} by gaining an angular resolution down to \SI{0.07}{\degree}. The typical energy range of primary gamma rays that can be observed with these IACTs is $\sim\SI{50}{\giga\electronvolt}$ up to $\sim\SI{10}{\tera\electronvolt}$.~\cite{iacts:magic}. Established IACT experiments are for instance \textit{H.E.S.S.} (Namibia)~\cite{iacts:hess}, \textit{VERITAS} (Arizona)~\cite{iacts:veritas}, and \textit{MAGIC} (La Palma)~\cite{iacts:magic}. The \text{Cherenkov Telescope Array} (\text{CTA}) will be a next-generation IACT array operating at two sites at the northern and southern hemisphere. In total, \num{118} telescopes are planned with three different sizes of \SI{4}{\meter}, \SI{12}{\meter}, \SI{23}{\meter}. It will extend the observable energy range up to $\sim\SI{100}{\tera\electronvolt}$.~\cite{iacts:cta} An artwork of a possible array is shown in figure~\ref{cta_artwork}.

\begin{figure}[H]
	\centering
	\includegraphics[width=0.9\textwidth]{eso_cta.jpg}
	\caption[Artwork of the proposed CTA site at ESO's Paranal Observatory]{\textbf{Artwork of the proposed CTA site at ESO's Paranal Observatory.}~\cite{iacts:cta_artwork} In this picture, the three classes of IACTs planned are rendered. This CTA site is located at the Paranal Observatory (Atacama desert, Chile) at an altitude of \SI{2100}{\meter} a.s.l. administrated by the European Southern Observatory (ESO). The final array layout may be different.~\cite{iacts:cta}}
	\label{cta_artwork}
\end{figure}

Even small IACTs can improve the energy measurement of other ground-based particle detectors since the Cherenkov light yield is proportional to the primary particle energy. Hybrid measurements of IACTs and particle detectors can enhance the energy resolution as recent studies show~\cite{iacts:extension}.

In contrast to large IACTs where the camera usually consists of an array of photomultiplier tubes (PMTs), smaller IACTs are based on Geiger-mode avalanche photo diodes (G-APDs) section or SiPMs which are linkages of many G-APDs. These devices are discussed in more detail in section~\ref{sec:sipm:working_principle}. In 2011, the \textbf{F}irst G-\textbf{A}PD \textbf{C}herenkov \textbf{T}elescope FACT started taking data with G-APDs as the first Cherenkov telescope of that kind~\cite{iacts:fact}.

\section{The \icecube Air-Cherenkov Telescopes \iceact}\label{sec:iceact_intro}

The \textit{\textbf{Ice}Cube \textbf{A}ir \textbf{C}herenkov \textbf{T}elescopes} \textit{\iceact} are one of the proposed extensions of the surface detector \icetop within the scope of \icecube-Gen2. It is planned to be an array of small air-Cherenkov telescopes which are compact and robust to stand the harsh weather conditions at South Pole.\\

A major purpose of \iceact is to enlarge the veto capabilities of \icetop by reducing the energy threshold down to about \SI{30}{\tera\electronvolt}~\cite{icecube:iceact}. It can also be useful for calibration of \icecube and \icetop and is even able to do cosmic-ray composition measurements, especially together with \icecube and \icetop via different detection channels~\cite{iceact:composition}.\\

The baseline design of \iceact is adapted from the \textit{FAMOUS}\footnote{\textbf{F}irst \textbf{A}uger \textbf{M}ulti-pixel photon counter camera for the \textbf{O}bservation of \textbf{U}ltra-high-energy air \textbf{S}howers} fluorescence telescope designed for the Pierre Auger Observatory in Argentina. It utilizes silicon photomultipliers (SiPMs, cf. section~\ref{sec:sipm:working_principle}) rather than photomultiplier tubes (PMTs) for the detection of fluorescence light.~\cite{famous:telescope} Since Cherenkov and fluorescence light range in similar wavelength regimes, Cherenkov light detection is possible as well. In addition, a third derivative of FAMOUS -- namely \textit{HAWC's Eye} -- has operated together with the HAWC gamma ray observatory. Studies showed, that a hybrid measurement of HAWC and HAWC's Eye could improve the energy resolution significantly~\cite{hawcseye:merlin}.\\

The \iceact camera consists of \num{61} SiPMs (cf. section~\ref{iceact:model:camera}). In 2015, a telescope demonstrator (called \textit{IceACT-2016}) with 7 pixels was deployed on the \icecube Laboratory (ICL) and has been operated in 2016. Analyses of the taken data have proven the concept to be successful~\cite{iceact:erik}. The figure~\ref{iceact:picture} shows the telescope the demonstrator was replaced with in January 2017.

\begin{figure}[H]
	\centering
	\includegraphics[width=0.9\textwidth]{IceActDemonstrator.jpg}
	\caption[The \iceact telescope on top of the \icecube Laboratory (ICL)]{\textbf{The \iceact telescope on top of the \icecube Laboratory (ICL).}~\cite{iceact:picture} The telescope seen in the picture was deployed in 2017 as a successor of the \iceact 7-pixel demonstrator.}
	\label{iceact:picture}
\end{figure}

In this thesis, the focus in on the development of a full telescope simulation with a subsequent parameterization. Thus, the technical properties of the \iceact optics are described in more detail associated with the simulation model in chapter~\ref{chap:iceact_model}. Nevertheless, figure~\ref{iceact:sketch} shows an overview sketch of the \iceact telescope.

\begin{figure}[H]
	\centering
	\includegraphics[width=0.6\textwidth]{IceActSketch.pdf}
	\caption[Sketch of the \iceact telescope]{\textbf{Sketch of the \iceact telescope.}~\cite{iceact:erik} The mechanical design of \iceact is based on a carbon tube on a box where the electronics is stored. On top of the tube, a Fresnel lens and a glass plate are fixed. At the focal distance of the lens, the \num{61} pixel camera is placed. The components are described in more detail in chapter~\ref{chap:iceact_model}.}
	\label{iceact:sketch}
\end{figure}