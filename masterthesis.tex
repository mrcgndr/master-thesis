% !TeX spellcheck = en_US
% Dokument
\documentclass[12pt,twoside,paper=a4,bibliography=totoc, listof=totoc,headsepline,footsepline,chapterprefix=true,open=right,headings=twolinechapter,headings=big,headings=standardclasses]{scrbook}
% -----------------------------------
% Sprache und Codierung
\usepackage[utf8]{inputenc}
\usepackage[english]{babel}
\usepackage[T1]{fontenc}
\usepackage{csquotes}
% Schriftart
\usepackage{mathpazo}
% für große Kapitelzahlen
\usepackage{anyfontsize}
%\setkomafont{chapter}{\normalfont\Huge}
\renewcommand*{\chapterheadstartvskip}{\vspace*{4\baselineskip}}
\renewcommand*{\chapterheadendvskip}{\vspace*{2\baselineskip}}
\renewcommand*{\chapterformat}{%
	{\fontsize{20}{30}\scshape\chapappifchapterprefix{}}%
	\fontsize{90}{30}\selectfont\rlap{\thechapter\autodot}%
}
\renewcommand*{\raggedchapter}{\raggedright}

\renewcommand*\othersectionlevelsformat[3]{%
	\llap{#3\autodot\enskip}%
}
% Bibliographie
\usepackage[backend=biber,style=numeric-comp,sorting=none,backref=true,sortcites=true]{biblatex}
\addbibresource{bibliography.bib}	
\renewcommand*{\bibfont}{\small}
\renewcommand*{\mkbibnamefamily}[1]{\textsc{#1}}
\renewcommand*{\mkbibnameprefix}[1]{\textsc{#1}}
\renewcommand*{\mkbibnamegiven}[1]{\textsc{#1}}
\usepackage{ragged2e}
\AtBeginBibliography{\RaggedRight}
\makeatletter
\def\blx@maxline{77}
% für URLs in Quellen und im Text
\makeatother	
\usepackage{url}		
\usepackage{mdwlist}
% Kopf- und Fußzeilen
\usepackage{scrlayer-scrpage}
\ofoot*{\raisebox{-2mm}{\pagemark}}
\renewcommand*{\chapterpagestyle}{scrheadings}
%\renewcommand*{\emptypagestyle}{scrheadings}
% Fußnoten nicht auf mehrere Seiten aufteilen
\interfootnotelinepenalty=10000
\usepackage[hang]{footmisc}
\setlength{\footnotemargin}{1em}
% Absatzanfang nicht einrücken
\setlength{\parindent}{0pt}
% Grafiken
\usepackage{graphicx}
\usepackage{pdfpages}
\graphicspath{{./images/}{./plots/}{./sketches/}}
\usepackage[space]{grffile}
% Tabellen
\usepackage{tabularx}
\usepackage{longtable}
\usepackage{multirow}
\usepackage{listings}
\usepackage{booktabs}
% Bildunterschriften
\usepackage{float}
\usepackage[format=plain, indention=1.5em, labelfont={bf}, justification=justified]{caption}
% Bilder neben Text
\usepackage{wrapfig}
% Zeileneinstellung
\usepackage{setspace} 
% Mathe		
\usepackage{amsmath}
\usepackage{amsfonts}
\usepackage{amssymb}
\usepackage{mathtools}
\DeclareMathOperator{\arctantwo}{arctan2}
% Counter
\usepackage{chngcntr}
\counterwithout{footnote}{chapter}
% Gibt an, wie viele Ebenen nummeriert werden sollen
\setcounter{secnumdepth}{5}
\setcounter{tocdepth}{5}
% Serifenschrift
\rmfamily
% Seitenränder
\usepackage[outer=22mm, inner=26mm, top=32mm, bottom=36mm]{geometry}
% Hyperlinks
\usepackage{hyperref}
% Todonotes
\usepackage{todonotes}
% Zitate
\usepackage{epigraph}
% Einheiten
\usepackage[per-mode=reciprocal, separate-uncertainty=true, binary-units=true, exponent-product = \cdot]{siunitx}
\DeclareSIUnit{\au}{a.u.}
%\sisetup{detect-family}
%\sisetup{range-phrase={\text{~to~}}}
% für Subcaptions
\usepackage{subcaption}
% mehere Referenzen in einem
\usepackage{cleveref}
% Umgebung für Variablenbeschreibung
\newenvironment{vardescription}
{\par\vspace{\abovedisplayskip}\noindent
	\tabularx{\columnwidth}{>{$}l<{$} @{${}\quad{}$} >{\raggedright\arraybackslash}X}}
{\endtabularx\par\vspace{\belowdisplayskip}}
\usepackage{xspace}
% Um zwei Subfigures aneinander zentriert auszurichten
\newsavebox{\savedimage}
\newcommand{\saveimageheight}[2][]{%
	\savebox{\savedimage}{\includegraphics[#1]{#2}}}
\newcommand{\raiseimage}[2][]{%
	\raisebox{.5\dimexpr\ht\savedimage-\height}{%
		\includegraphics[#1]{#2}}}%
% Zeilenumbruch nach Paragraph und Subparagraph
\RedeclareSectionCommands[afterskip=1sp]{paragraph,subparagraph}
% Farbe von Überschriften
\usepackage{xcolor}
\addtokomafont{chapter}{\color[cmyk]{1,0.6,0,0.5}}
\addtokomafont{section}{\color[cmyk]{1,0.6,0,0.33}}
\addtokomafont{subsection}{\color[cmyk]{1,0.54,0,0.22}}
\addtokomafont{subsubsection}{\color[cmyk]{1,0.6,0,0}}
\addtokomafont{paragraph}{\color[cmyk]{0.67,0.4,0,0}}
\addtokomafont{subparagraph}{\color[cmyk]{0.5,0.3,0,0}}
% Geant4
\newcommand{\geant}{\textsc{Geant4}\xspace}
% IceAct, IceCube, IceTop
\newcommand{\iceact}{\mbox{IceAct}\xspace}
\newcommand{\icecube}{\mbox{IceCube}\xspace}
\newcommand{\icetop}{\mbox{IceTop}\xspace}
 
% Document
% -----------------------------------
\begin{document}

\frontmatter 
	% !TeX spellcheck = en_US
\begin{titlepage}
\addtolength{\oddsidemargin}{4mm}
\begin{center}

% Titel der Arbeit 
\LARGE
\textbf{Simulation of the Optics of the Imaging Air Cherenkov Telescope IceAct with Geant4} \\[15mm]
% Simulation der Optik des abbildenden Luft Tscherenkow Teleskops IceAct mit Geant4

% Autor
{\large by}\\[1mm]
\Large
Maurice Günder, B.Sc.\\[18mm]

% Art der Arbeit
Master's Thesis in Physics\\[70mm]

{\large in}\\[1mm]
April 2019\\[35mm]

Univ.-Prof. Dr. Christopher Wiebusch \\\textsc{III. Physikalisches Institut B \\Rheinisch- Westfälische Technische Hochschule Aachen}

\end{center}
\end{titlepage} 		% Titelblatt
    % !TeX spellcheck = en_US
\cleardoublepage

\vspace*{15cm}

\begin{flushleft}
	\textbf{First Referee}\\
	Univ.-Prof. Dr. Christopher Wiebusch\\
	Physics Institute III B\\
	RWTH Aachen University\\
	\bigskip
	
	\textbf{Second Referee}\\
	Prof. Dr. Thomas Bretz\\
	Physics Institute III A\\
	RWTH Aachen University
	\bigskip
	
	\textbf{Supervisor}\\
	Dr. Jan Auffenberg\\
	Physics Institute III B\\
	RWTH Aachen University\\
\end{flushleft}		% Betreuer, Korrektor
    \protect\thispagestyle{scrheadings}
    \onehalfspacing             % Zeilenabstand ab hier 1.5 zeilig
	%\renewcommand{\contentsname}{Content}
	\tableofcontents			% Inhaltsverzeichnis

% -----------------------------------
\mainmatter 					% die einzelnen Kapitel
    \protect\pagestyle{scrheadings}
    % !TeX spellcheck = en_US
\chapter{Introduction}

\section{Motivation}

\section{The IceCube Neutrino Observatory}

Since January 2011, the IceCube Neutrino Observatory at the South Pole is measuring neutrinos emanating from various sources. For this purpose a detector instrumented with digital optical modules (\enquote{DOMs}) is installed deep in the antarctic ice. 5160 of these optical sensors are arranged on 86 strings at a height between \SI{1450}{\meter} and \SI{2450}{\meter} below the surface. The central region of this In-Ice Array which has a higher density of DOMs is called \enquote{DeepCore}.

\begin{figure}[h]
	\includegraphics[width=\textwidth]{IceCubeDetector.pdf}
	\caption[Schematic view of IceCube]{\textbf{Schematic view of the IceCube Neutrino Observatory. \cite{icecube:instrumentation}} The in-ice array with the denser sub-array DeepCore as well as the surface array IceTop is sketched. Different station colors represent different deployment stages.}
\end{figure}

Neutrinos are very interesting elementary particles because of their weak interaction cross section and their electrical neutrality. This fact makes it possible for neutrinos to let them point back to their sources which is exploited in the search for astrophysical processes like active galactic nuclei, supernovae, or gamma-ray bursts. Since they are able to reach us without scattering processes, neutrinos can even give information about sources at cosmological distances. Simultaneously, the weak interaction potential is what neutrino detection makes challenging. Therefore, a detector with a large scale active volume is needed. In the case of IceCube, this is \SI{1}{\cubic\kilo\meter} of ice.

At the surface on top of the in-ice detector the cosmic ray air shower array IceTop is installed to detect Cherenkov radiation (see \ref{sec:cherenkov}). IceTop consists of 81 stations approximately arranged in the same grid as the in-ice strings. Each station has two tanks filled up with ice and two standard IceCube DOMs. This arrangement makes it possible for IceTop to detect primary cosmic rays (see \ref{sec:cosmicrays}) in the energy range of \si{\peta\electronvolt} to \si{\exa\electronvolt}. \cite{icecube:instrumentation}

\section{Air Showers and Cosmic Rays}\label{sec:cosmicrays}

\section{Atmospheric Cherenkov Light}\label{sec:cherenkov}

\section{Imaging Air Cherenkov Telescopes}

\subsection{Imaging Technique}

\subsection{Photon Detection}

\subsection{IceAct}
    % !TeX spellcheck = en_US
\chapter{The IceAct Model in \geant}

\section{\geant}
\geant is a multi-purpose simulation framework for the passage of particles trough matter, written in \textit{C++} and developed by the \geant Collaboration at CERN. It includes physics models, geometry, tracking, hits, and digitization and thus allows a detailed simulation and response analysis for particle detectors in many application fields like particle and accelerator physics, space engineering or medical science. In the framework's source some basic and advanced use cases are implemented and provided as examples. The toolkit is built up of multiple categories (or modules) using each other (cf. figure \ref{geant4:categories}). \cite{geant4}

\begin{wrapfigure}{r}{0.5\textwidth}
	\centering
	\includegraphics[width=0.5\textwidth]{Geant4Logo.png}
	\caption[\geant Logo]{\textbf{\geant logo.} \cite{geant4:logo}}	
\end{wrapfigure}

\begin{figure}[h]
	\centering
	\includegraphics[width=0.6\textwidth]{Geant4ConceptDiagram.pdf}
	\caption[\geant category diagram]{\textbf{Diagram of relationships between \geant categories. \cite{geant4}} The circles represent a \enquote{using} relation. The category with the circle next to the box uses the linked one.}	
	\label{geant4:categories}
\end{figure}

Especially for the use case of IceAct \geant is capable of simulating Cherenkov (optical) photons, material properties like transmission, reflection, and refraction, as well as detection efficiency properties of the SiPMs.

Since this thesis is about an approach of an all-encompassing telescope simulation, \geant provides all major possibilities to get a distinct analysis of the entire optical system of IceAct.

\subsection{FAMOUS telescope simulation}
The fluorescence telescope FAMOUS\footnote{\textbf{F}irst \textbf{A}uger \textbf{M}ulti-pixel
photon counter camera for the \textbf{O}bservation of \textbf{U}ltra-high-energy air \textbf{S}howers} for the Pierre Auger Observatory in Argentina is developed at RWTH Aachen to measure fluorescence light originating from ultra-high-energy cosmic rays (UHECR) by using Silicon Photomultipliers (SiPMs). Withing the development, a detailed \geant simulation has been elaborated. \cite{famous:sim_github,famous:sim_github} The telescope design of FAMOUS is very similar since the detection technique and the optics system is basically the same. Therefore, the IceAct telescope simulation is heavily based on this FAMOUS \geant framework. A detailed discourse and a summary of previous analyses can be found in \cite{famous:niggemann}.

\section{Materials}

For an optical device, the material that the light should pass has to be chosen deliberately. Especially the transmission properties, processability, and for IceAct in particular the resistance against harsh weather conditions are of interest.

The glass plate on top of IceAct is made of SCHOTT BOROFLOAT\textsuperscript{\textregistered} 33 borosilicate glass with a thickness of \SI{2+-0.2}{\milli\meter} and a diameter of \SI{650.3+-1}{\milli\meter}. Borosilicate is chosen for its high durability, transparency in the interesting spectral region, flatness, and weak fluorescence intensities. The refractive index is evaluated at some wavelength. Since we need to have a full dispersion relation the points are spline interpolated (cf. orange curve in figure \ref{iceact:model:material:refractive_index}). \cite{iceact:borosilicate:datasheet}. 

In the data sheet \cite{iceact:borosilicate:datasheet} the transmission properties are given for a vertical light and a glass plate of a thickness of $d = \SI{6.5}{\milli\meter}$. Therefore, the transmission curve $T_\text{total}(\lambda)$ includes the internal absorption as well as the two interface transitions into and out of the borosilicate.
\begin{align}
	T_\text{total}(\lambda) = T_\text{interface}^2(\lambda)\cdot T_\text{internal}(d=\SI{6.5}{\milli\meter},\lambda)
	\label{eq:transmission}
\end{align}
The transmission at the interface can be calculated by using the Fresnel equations. In case of perpendicular light, it is
\begin{align}
	T_\text{interface}(\lambda) = 1 - \left(\frac{n(\lambda)-n_\text{air}}{n(\lambda)+n_\text{air}}\right)^2
	\label{eq:perp_interface_transmission}
\end{align}
In \geant the wavelength dependent absorption length $a(\lambda)$ has to be implemented which is given by the exponential absorption law
\begin{align}
	I(x) = I_0 e^{-\frac{x}{a}} \Leftrightarrow a = - \frac{x}{\ln\left(\frac{I(x)}{I_0}\right)}
	\label{eq:absorptionlaw}
\end{align}
Thus, one gets the absorption length by using equations \eqref{eq:transmission}, \eqref{eq:perp_interface_transmission}, and \eqref{eq:absorptionlaw}.
\begin{align}
	a(\lambda) &= - \frac{d}{\ln T_\text{internal}(d,\lambda)}\\
	&= -\frac{d}{\ln T_\text{total}(\lambda)-2\ln\left(1 - \left(\frac{n(\lambda)-n_\text{air}}{n(\lambda)+n_\text{air}}\right)^2\right)}\,,
\end{align}
which is implemented in \geant material properties with $n_\text{air} = 1$. Figure \ref{iceact:model:material:transmission} shows the three transmission components as orange lines.

The Fresnel lens and the Winston Cones in IceAct are made of polymethyl methacrylate (PMMA, acrylic, or plexiglass). The dispersion $n(\lambda)$ can be parametrized with the empirical \textit{Sellmeier equation}. For glasses the usual form is
\begin{align}
	n^2(\lambda) = 1 + \frac{B_1\lambda^2}{\lambda^2-C_1} + \frac{B_2\lambda^2}{\lambda^2-C_2} + \frac{B_3\lambda^2}{\lambda^2-C_3}\,,
	\label{eq:sellmeier}
\end{align}
with the \textit{Sellmeier coefficients} $B_{1,2,3}$ and $C_{1,2,3}$. \cite{iceact:sellmeier} Table \ref{iceact:model:pmma_sellmeiercoeffs} shows the used coefficients and the function is plotted in figure \ref{iceact:model:material:refractive_index} (blue curve).

\begin{table}[h]
	\centering
	\begin{tabular}{c|l}
		$B_1$  & \num{0.99654}  \\
		$B_2$  & \num{0.18964}  \\
		$B_3$  & \num{0.00411}  \\
		$C_1$  & \SI{0.00787}{\micro\meter\squared}  \\
		$C_2$  & \SI{0.02191}{\micro\meter\squared}  \\
		$C_3$  & \SI{3.85727}{\micro\meter\squared}  \\
	\end{tabular}
	\caption[Sellmeier coefficients for PMMA]{\textbf{Sellmeier coefficients for PMMA.} \cite{iceact:refractiveindex} The above-mentioned coefficients are used in the \geant material properties for PMMA. The related Sellmeier equation \eqref{eq:sellmeier} is plotted in figure \ref{iceact:model:material:refractive_index} as the blue curve.}
	\label{iceact:model:pmma_sellmeiercoeffs}
\end{table}

For the transmission properties of PMMA, the same method as for borosilicate is used (see above). Therefore, the data stated in \cite{famous:niggemann} is taken as $T_\text{internal}(d = \SI{3}{\milli\meter})$. Figure \ref{iceact:model:material:transmission} shows the three transmission components as blue lines.

\begin{figure}[h]
	\centering
	\includegraphics[width=0.7\textwidth]{material/transmission.pdf}
	\caption[Transmission of used materials]{\textbf{Transmission functions of materials used in the simulation.} The total transmission function is the product of internal and two interface transmissions which is evaluated for a perpendicularly incident particle in this plot. Thus, the solid lines represent a complete (perpendicular) transition through a \SI{3}{\milli\meter} thick layer of the respective material. For a better comparison, the data of internal transmission for borosilicate is converted from \SI{6.5}{\milli\meter} into \SI{3}{\milli\meter}.}
	\label{iceact:model:material:transmission}	
\end{figure}

\begin{figure}[h]
	\centering
	\includegraphics[width=0.7\textwidth]{material/refractive_index.pdf}
	\caption[Refractive index of used materials]{\textbf{Refractive index of materials used in the simulation.} For PMMA the dispersion is calculated by evaluating the Sellmeier equation introduced in this section. The refractive index for the used borosilicate is only given at specific wavelengths. \cite{iceact:borosilicate:datasheet} Therefore, spline interpolation is used to reconstruct the full curve.}
	\label{iceact:model:material:refractive_index}	
\end{figure}

The tube, back plane and other coating surfaces are simulated as \enquote{dummy} material with no reflection or transmission parameters. A particle that hits those surfaces is absorbed and not considered any further.

\section{Optics}

In the following section the four optical components of the IceAct \geant model are discussed. To have a first glimpse of the model, see figure \ref{iceact:model:cut}.
\begin{figure}[h]
	\centering
	\includegraphics[width=0.6\textwidth]{IceActGeant4Model.pdf}
	\caption[IceAct \geant model]{\textbf{The IceAct \geant model.} Cross-sectional sketch of the IceAct optics in \geant with all simulated components. They are described in detail in \cref{iceact:model:fresnellens,iceact:model:camera,iceact:model:glassplate}.}
	\label{iceact:model:cut}	
\end{figure}

\subsection{Glass Plate}\label{iceact:model:glassplate}

\todo{glass plate}

\subsection{Fresnel Lens}\label{iceact:model:fresnellens}

\begin{figure}[h]
	\centering
	\includegraphics[width=0.5\textwidth]{FresnelVsNormalLens.pdf}
	\caption[Comparison conventional vs. Fresnel lens]{\textbf{Comparison between a conventional \enquote{thick} lens and a Fresnel lens. \cite{famous:eichler}} For the functionality of a lens the radius-dependent sagitta function $z(\rho)$ is crucial. To get rid of the bulky material of a conventional lens, the Fresnel lens is divided into annular \enquote{prisms} called \enquote{grooves}. The slope angle $\delta$ of each groove is a local approximation of the sagitta function to ensure the imaging capability.}
	\label{iceact:model:fresnelvsthick}	
\end{figure}

\begin{figure}[h]
	\centering
	\includegraphics[width=0.5\textwidth]{FresnelGroove.pdf}
	\caption[Fresnel groove]{\textbf{Cross-sectional sketch of a Fresnel groove. \cite{famous:eichler}}}
	\label{iceact:model:fresnelgroove}	
\end{figure}

\todo{write}
\cite{famous:eichler}

\subsection{Camera}\label{iceact:model:camera}

\todo{write}

\subsubsection{Winston Cones}

\todo{write}

\subsubsection{Silicon Photomultiplier}

\todo{write}


    % !TeX spellcheck = en_US
\chapter{\iceact Simulation}\label{chap:iceact_sim}

This chapter will discuss the \geant simulation process. \todo{bisschen mehr?}

\section{Simulation of Single Components}

In order to make some first checks and optimization, the focus is on single essential components, i.e. the Fresnel lens and the Winston cones.

\subsection{\enquote{Best} Wavelength}\label{sec:best_wvl}

Many of the \iceact telescope properties have (non-linear) wavelength dependencies (cf. section~\ref{sec:iceact:model:material}). Additionally, the Cherenkov spectrum is wavelength dependent as well (cf. figure~\ref{airshowers:cherenkovspectrum}). By implication, there has to be a wavelength $\lambda^\ast$ that \iceact is the most efficient. One can determine $\lambda^\ast$ by looking at the following limiting functions.

\begin{itemize}
	\item The Cherenkov spectrum. The data shown in figure~\ref{airshowers:cherenkovspectrum} (La Palma, \SI{2200}{\meter} a.s.l.) is chosen.
	\item The internal transmission function of PMMA. It is assumed that a photon has to pass approximately \SI{30}{\milli\meter} of PMMA to get to the SiPMs. Thus, the internal transmission function shown in figure~\ref{iceact:model:material:transmission} as blue dotted-dashed line has to be exponentiated by \num{10} to hold for this case.
	\item The internal transmission function of borosilicate, i.e. the material of the glass plate. The photons have to pass approximately \SI{2}{\milli\meter}. Exponentiation of a factor $\frac{2}{3}$ of the orange dotted-dashed line in figure~\ref{iceact:model:material:transmission} leads to the desired function.
	\item The photon detection efficiency (PDE) function of the SiPMs interpolated for $V_\text{OV} = \SI{5}{\volt}$ (cf. orange curve in figure~\ref{sipm:pde}).
\end{itemize}

All of these functions are normalized, i.e. divided by their own maximum, and than multiplied which results in a new (relative) efficiency function. The maximum of this function again is the \enquote{best} wavelength found to be $\lambda^\ast = \SI{411}{\nano\meter}$. Figure~\ref{best_wvl} shows the procedure graphically. 

\begin{figure}[H]
	\centering
	\includegraphics[width=0.7\textwidth]{best_wvl.pdf}
	\caption[\enquote{Best} wavelength]{\textbf{\enquote{Best} wavelength.} All limiting functions (Cherenkov spectrum, matrial transmission curves, and photon detection efficiency) are nomalized to each maximum and multiplied. The maximum of the product is defined to be the \enquote{best}, i.e. most efficient, wavelength $\lambda^\ast=\SI{411}{\nano\meter}$.}
	\label{best_wvl}
\end{figure}

\subsection{Focal Plane Shift}\label{sec:focalplaneshift}

So there is a most efficient wavelength for the \iceact telescope as shown in the last section~\ref{sec:best_wvl}. As stated in section~\ref{iceact:model:fresnellens} and in the ORAFOL data sheet~\cite{iceact:fresnellens:datasheet}, the focal distance of the Fresnel lens $z_f=\SI{502.1}{\milli\meter}$ is given for a certain wavelength $\lambda=\SI{546+-27.3}{\nano\meter}$. Thus, the focal distance at $\lambda^\ast=\SI{411}{\nano\meter}$ may be different which gives a possibility for potential improvement for the optical properties of \iceact. To investigate a shift of the focal plane, one has to define a quantity to optimize. In this simulation, the aberration radius $r_{90}$ is used for this purpose (cf. section~\ref{iceact:model:fresnellens}). A point spread function measurement of the Fresnel lens for monochromatic light with $\lambda=\SI{546}{\nano\meter}$ and different incidence angles $\theta$ is done in~\cite{famous:niggemann} by using ray tracing simulation. In this thesis, a PSF simulation is done as well but for wavelengths between $\SI{270}{\nano\meter}$ and $\SI{900}{\nano\meter}$ and vertical incidence $\theta=\SI{0}{\degree}$. The focal plane is fixed at the focal distance $z_f=\SI{502.1}{\milli\meter}$. In total, a vertical beam of \num{1e8} photons with uniformly density and uniformly distributed wavelengths in the interval given above is simulated. On the focal plane, the wavelength $\lambda_\text{hit}$, position $(x_\text{hit},y_\text{hit}))$, and angle $(\theta_\text{hit},\phi_\text{hit})$ of the detected photons are registered. The goal is to measure the minimal aberration radius at the suggested focal distance $z_f=\SI{502.1}{\milli\meter}$ and a possible focal plane shift in order to minimize the aberration radius for $\lambda^\ast=\SI{411}{\nano\meter}$.\\

For the first measurement, the aberration radius is evaluated by calculating the \SI{90}{\percent}-quantile of the distances $r_\text{hit}$ between the hit position and the optical axis\footnote{Normally, this is the centroid rather than the optical axis but in the case of parallel light, the centroid is assumed to be at $(x,y) = (0,0)$.} given by
\begin{align}
	r_\text{hit} = \sqrt{\left(x_\text{hit}-x_\text{centroid}\right)^2+\left(y_\text{hit}-y_\text{centroid}\right)^2} \overset{(x,y)_\text{centroid}=(0,0)}{=}\sqrt{x_\text{hit}^2+y_\text{hit}^2}\,.
\end{align}
By doing this for small wavelength ranges, one gets a wavelength-dependent aberration radius $r_{90}(\lambda)$ on the focal plane. As a result, a minimal aberration radius of \SI{1.78}{\milli\meter} is reached at a wavelength of \SI{403}{\nano\meter}. Figure~\ref{psf_at_focal_plane} shows $r_{90}(\lambda)$.\\

\begin{figure}[H]
	\centering
	\includegraphics[width=0.8\textwidth]{focalplaneshift/psf_r90.pdf}
	\caption[Aberration radius on the focal plane]{\textbf{Aberration radius on the focal plane.} $r_{90}$ is calculated for vertical light as a function of the wavelength. The focal distance is fixed to $z_f=\SI{502.1}{\milli\meter}$. The minimal aberration radius of \SI{1.78}{\milli\meter} is reached at a wavelength of \SI{403}{\nano\meter}.}
	\label{psf_at_focal_plane}
\end{figure}

For the focal plane shift, one considers a small wavelength range and again calculates the aberration radius. Since the incidence angles on the focal plane are known, one can calculate the point of incidence for a hypothetical focal plane at a position of $z_f+\Delta z$, where $\Delta z$ is the focal plane shift. A trigonometrical approach gives
\begin{align}
	r_\text{hit}(\Delta z) = \sqrt{\left(x_\text{hit}-\Delta z\tan\theta_\text{hit}\cos\phi_\text{hit}\right)^2 + \left(y_\text{hit}-\Delta z\tan\theta_\text{hit}\sin\phi_\text{hit}\right)^2}\,.
\end{align}
Thus, the aberration radius can be calculated for each focal plane shift $\Delta z$ and wavelength $\lambda$. Than, an optimal focal plane shift can be found by minimizing the aberration radius. Figure~\ref{focalplaneshift} shows this calculation for different wavelengths by evaluation the beam \enquote{caustic}\footnote{In this context, caustic means the photon density distribution along the optical axis. Usually in beam optics, a caustic just describes the envelope of the beam.} for different focal plane shifts. One can clearly see that the focal length increases with wavelength. Additionally, figure~\ref{focalplaneshift_zoomout} shows a zoomed-out version of figure~\ref{focalplaneshift_bestwvl}. In particular for the most efficient wavelength $\lambda^\ast=\SI{411}{\nano\meter}$, a resulting marginal focal plane shift of $\Delta z=\SI{1.25}{\milli\meter}$ shows, that the standard focal distance is already quite good for \iceact. Nevertheless, the focal plane shift is considered in the final parameterization simulation.

\begin{figure}[H]
	\centering
	\begin{subfigure}[t]{0.48\textwidth}
		\centering
		\includegraphics[width=\textwidth]{focalplaneshift/caustic_300nm.png}
		\subcaption{$\lambda=\SI{300+-1}{\nano\meter}$}
	\end{subfigure}
	\hfill
	\begin{subfigure}[t]{0.48\textwidth}
		\centering
		\includegraphics[width=\textwidth]{focalplaneshift/caustic_411nm.png}
		\subcaption{$\lambda^\ast=\SI{411+-1}{\nano\meter}$}
		\label{focalplaneshift_bestwvl}
	\end{subfigure}
	\hfill
	\begin{subfigure}[t]{0.48\textwidth}
		\centering
		\includegraphics[width=\textwidth]{focalplaneshift/caustic_546nm.png}
		\subcaption{$\lambda=\SI{546+-1}{\nano\meter}$}
		\label{focalplaneshift_famouswvl}
	\end{subfigure}
	\hfill
	\begin{subfigure}[t]{0.48\textwidth}
		\centering
		\includegraphics[width=\textwidth]{focalplaneshift/caustic_800nm.png}
		\subcaption{$\lambda=\SI{800+-1}{\nano\meter}$}
	\end{subfigure}
	\caption[\enquote{Caustic} histograms for the focal plane shift]{\textbf{\enquote{Caustic} histograms for the focal plane shift.} For focal plane shifts~$\Delta z$ between $\SI{-50}{\milli\meter}$ and $\SI{50}{\milli\meter}$, the incident positions of simulated photons are histrogramized by calculating their distance from the optical axis $r_\text{hit}(\Delta z)$. This results in a photon density plot called \enquote{caustic}. A focal plane shift of $\Delta z = 0$ is equivalent to the \enquote{standard} focal distance $z_f=\SI{502.1}{\milli\meter}$. A positive $\Delta z$ connotes a shift away from the lens. Thus, the lens is located on top of the shown plots. The calculation is done for different wavelengths, especially for the \enquote{best} wavelength in (\subref{focalplaneshift_bestwvl}) and for the wavelength which the focal distance is set for in (\subref{focalplaneshift_famouswvl}). The red line shows the aberration radius $r_{90}(\Delta z)$ and the focal plane shift where its minimum is reached marked by the red dashed line.}
	\label{focalplaneshift}
\end{figure}

\begin{figure}[H]
	\centering
	\includegraphics[width=0.8\textwidth]{focalplaneshift/caustic_411nm_zoomout.png}
	\caption[Zoomed-out caustic histogram]{\textbf{Zoomed-out caustic histogram.} The plot shows a zoomed-out version of figure~\ref{focalplaneshift_bestwvl}. The red band on the top denotes the location of the Fresnel lens. Besides a major focal spot at $\Delta z\approx\SI{0}{\milli\meter}$, one can see two minor ones at $\Delta z\approx\SI{-300}{\milli\meter}$ and $\Delta z\approx\SI{220}{\milli\meter}$ which come from the \enquote{false} refractions mentioned in section~\ref{iceact:model:fresnellens}.}
	\label{focalplaneshift_zoomout}
\end{figure}

\subsection{Winston Cone Simulation}



\section{Simulation Strategy and Verification}

\subsection{Winston Cone Meshing}\label{sec:wico_meshing}

\subsection{Aberration Effects -- \enquote{Ghost Image}}\label{sec:ghost_image}

\begin{figure}[H]
	\centering
	\includegraphics[width=0.9\textwidth]{GhostImage.pdf}
	\caption[Schematic sketch for the \enquote{ghost image} ray path]{\textbf{Schematic sketch for the \enquote{ghost image} ray path.} long caption}
	\label{ghostimage_path}
\end{figure}

\section{Parametrization Simulation}

\begin{figure}[H]
	\centering
	\saveimageheight[width=0.49\textwidth]{GeantCoords.pdf}
	\begin{subfigure}[t]{0.49\textwidth}
		\raiseimage[width=\textwidth]{CameraPixels.pdf}
		\subcaption{Top view of the \iceact camera with pixel numbering and coordinate system. The $z$-axis comes out of the drawing plane.}
	\end{subfigure}
	\hfill
	\begin{subfigure}[t]{0.49\textwidth}
		\usebox{\savedimage}
		\subcaption{Coordinate system for the \iceact telescope. The tube is sketched as a the blue cylinder. The coordinate origin is set as the center of the focal plane on the Winston cones' entrance windows. An exemplary incoming photon drawn as red line impinges on the glass plate under a zenith angle $\theta$ and an azimuth angle $\phi$.}
	\end{subfigure}
	\caption[Coordinate system used in \geant and for the simulation]{\textbf{Coordinate system used in \geant and for the simulation.} }
	\label{geant_coords}
\end{figure}
    % !TeX spellcheck = en_US
\chapter{IceAct Parameterization}

The simulation results discussed in chapter \ref{chap:simresults} can now be used to parameterize the telescope response of IceAct. The major goal of this is to provide a fast way to evaluate the detection probability of incident photons in each camera pixel which is done by elaboration of a lookup table (\textit{LUT}). In the following sections, the method of setting up this LUT is described.

\section{Kernel Density Estimation}

Kernel density estimation \textit{KDE} is a non-parametric method to estimate a probability density function of a random variable by a given finite data sample. The standard \textit{kernel density estimator}
\begin{align}
	\hat{f}(x)=\frac{1}{nh}\sum_{i=1}^{n}K\left(\frac{x-X_i}{h}\right)\,,
	\label{eq:kde}
\end{align}
is the sum of \textit{kernel functions} $K(\dots)$ for each data point $X_i$. The non-negative parameter $h$ is the \textit{bandwidth} and is a measure for the smoothing of the resulting KDE: the KDE gets smoother with increasing $h$. Due to the normalization factor, $\hat{f}(x)$ is normed to
\begin{align}
	\int\limits_{-\infty}^{+\infty}\hat{f}(x)dx \equiv 1\,.
\end{align}

\begin{figure}[h]
	\centering
	\includegraphics[width=0.5\textwidth]{KDE_example.pdf}
	\caption[Example KDE with different smoothing]{\textbf{Example KDE with different smoothing.} \cite{kde:example_plot} Kernel density estimation is applied on a data sample with 100 random numbers drawn from a normal distribution (gray curve). The blue, green, and orange curves have different bandwidths.}
	\label{kde:example_1d}	
\end{figure}

The kernel function can be a very distinct function that can in principle describe any probability density. In this parameterization method a Gaussian kernel
\begin{align}
	K(t) = \frac{1}{\sqrt{2\pi}}e^{-\frac{1}{2}t^2}
\end{align}
is used.

In order to get the KDE describing the probability density appropriately, one has to choose a reasonable bandwith for the given data sample size. As shown in figure \ref{kde:example_1d}, too low bandwith results in a spiky, fluctuating KDE. If the bandwidth is too high, one might get a rather inaccurate estimator. Therefore it seems reasonable to choose different bandwidth in regions with different amount of statistics which then is called \textit{adaptive} kernel density estimation. \cite{kde:schoenen, kde:wangwang}

\subsection{Adaptive KDE with Gaussian Kernel}
As one can see in the simulation results (cf. \todo{verweis zu map}) and in figure \ref{kde:example_scatter}, each pixel has a certain region of photon directions where it is efficient. However, each pixel is almost \enquote{blind} for most other directions. This results in statistically stable -- \enquote{dense} -- regions but also \enquote{sparse} regions dominated by scattered photons that undergo large statistical fluctuations. Thus, and based on the former conclusions, an approach to adapt the bandwidth to the local statistics should perform well. In this thesis, an algorithm presented by \textsc{B. Wang} and \textsc{X. Wang} in \cite{kde:wangwang} which has been implemented within the scope of \cite{kde:schoenen} is used. The adaptive (in principle weighted) kernel density estimator is calculated by \cite{kde:schoenen,kde:wangwang}
\begin{align}
	f(\vec{x}) = \sum_{i=0}^{n} \frac{w_i}{N_i}e^{-\frac{1}{2}(\vec{x}-\vec{X_i})^T \frac{1}{h\lambda_i} \mathbf{C}^{-1} (\vec{x}-\vec{X_i})}\,,
\end{align}
with
\begin{vardescription}
	n & total number of data points,\\
	w_i & weight of th $i$-th data point,\\
	N_i & Gaussian normalization,\\
	\vec{X_i} & coordinate vector of the $i$-the data point,\\
	h & global bandwidth factor,\\
	\lambda_i & local bandwidth factor,\\
	\mathbf{C} & covariance matrix.\\
\end{vardescription}
The global bandwidth factor $h$ is calculated by the \textit{Silverman rule} \cite{kde:schoenen,kde:wangwang}
\begin{align}
	h = \left(\frac{n(d+2)}{4}\right)^{-\frac{1}{d+4}}\,,
\end{align}
where $d$ equals the number of dimensions of the data (here $d=2$). The local bandwidth $\lambda_i$ is the factor where the local statistics of each data point comes in. It is defined as \cite{kde:schoenen,kde:wangwang}
\begin{align}
	\lambda_i = \left(\frac{\hat{f}(\vec{X_i})}{g}\right)^{-\alpha}\,,
\end{align}
with
\begin{vardescription}
	\hat{f}(\vec{X_i}) = f(\vec{X_i})|_{w_i=\lambda_i=1}\,,\\
	\ln{g} = n^{-1} \sum_{i=0}^{n} \ln{\hat{f}(\vec{X_i})}\,,\\
	\alpha\in[0,1]\,.
\end{vardescription}
For the IceAct parameterization the sensitivity parameter $\alpha$ is set to $\alpha=\num{0.3}$, since this has been shown to be an appropriate value as well as in \cite{kde:schoenen}. Additionally, all $n$ photon hits have the same weight, which results in \cite{kde:schoenen,kde:wangwang}
\begin{align}
	w_i = \frac{1}{n}\,.
\end{align}
To get an idea of the given data points for which the KDE should be calculated, figure \ref{kde:example_scatter} shows a scatter plot of detected photons for an arbitrary camera pixel in a possible range of wavelengths. The need for an adaptive KDE approach is clearly visible by regions with very different statistical densities. In addition, figure \ref{kde:comparison} shows strikingly the difference between an adaptive and a non-adaptive KDE.

\begin{figure}[h]
	\centering
	\includegraphics[width=0.5\textwidth]{scatter_px20_wvl410-420nm.png}
	\caption[Example: directions of detected photons as a scatter plot]{\textbf{Example: directions of detected photons as a scatter plot.} A subset of simulated photon directions that are detected in camera pixel 20 with wavelengths between \SI{410}{\nano\meter} and \SI{420}{\nano\meter} are shown in a polar plot. One can clearly see that there are regions with high and low statistical significance.}
	\label{kde:example_scatter}	
\end{figure}

\begin{figure}[h]
	\centering
	\includegraphics[width=\textwidth]{comparison_adaptive_nonadaptive.pdf}
	\caption[Comparison: adaptive vs. non-adaptive KDE]{\textbf{Comparison: adaptive vs. non-adaptive KDE.} The difference between an adaptive (left) and non-adaptive (right) KDE approach is visible. Especially in the region with low statistics, the non-adaptive KDE is dominated by the fluctuations.}
	\label{kde:comparison}	
\end{figure}

\subsection{Bootstrapping}

One problem of kernel density estimation is that it does not provide any statistical information, i.e. \enquote{how precise} the probability density is estimated.

\section{Detection Efficiency}

\subsection{The HEALPix Algorithm}

In order to account for the spherical shape of the angles of incidence it is useful to have angular bins with equal areas. Therefore, in this parameterization method, the \textit{HEALPix} (\textbf{H}ierarchical \textbf{E}qual \textbf{A}rea iso\textbf{L}atitude \textbf{Pix}elization) algorithm is used. It allows a uniform pixelization of incidence angles to the telescope.\\

\todo{healpix pixelization image}

The basic idea is to subdivide the sphere into 12 quadrilateral panes which can then further divided into more sub-panes as shown in figure \todo{figure reference}. A parameter $k$ numbers the subdivision step so that the number of sub-panes per each of the 12 panes is $N_{\text{side}}^2=\left(2^k\right)^2$. Hence, the total amount of pixels on the sphere is then
\begin{align}
N_\text{pix} = 12N_\text{side}^2\,.
\end{align}
Obviously, the angular resolution is dependent on $N_\text{pix}$, henceforth referred to as \enquote{pixel}.

\begin{table}[h]
\centering
\begin{tabular}{S[table-format=2.0]|S[table-format=4.0]|S[table-format=9.0]|S[table-format=2.2]}
\multicolumn{1}{l|}{$k$} & \multicolumn{1}{l|}{$N_\text{side} = 2^k$} & \multicolumn{1}{l|}{$N_\text{pix} = 12N_\text{nside}^2$} & \multicolumn{1}{l}{$\theta_\text{pix}$} \\
\hline
0  & 1    & 12        &  \SI{58.6}{\degree}\\
1  & 2    & 48        &  \SI{29.3}{\degree}\\
2  & 4    & 192       &  \SI{14.7}{\degree}\\
3  & 8    & 768 	  &  \SI{7.33}{\degree}\\
4  & 16   & 3072      &  \SI{3.66}{\degree}\\
5  & 32   & 12288     &  \SI{1.83}{\degree}\\
6  & 64   & 49152     &  \SI{55.0}{\arcminute}\\
7  & 128  & 196608    &  \SI{27.5}{\arcminute}\\
8  & 256  & 786432    &  \SI{13.7}{\arcminute}\\
9  & 512  & 3145728   &  \SI{6.87}{\arcminute}\\
10 & 1024 & 12582912  &  \SI{3.44}{\arcminute}\\
11 & 2048 & 50331648  &  \SI{1.72}{\arcminute}\\
12 & 4096 & 201326592 &  \SI{51.5}{\arcsecond}\\
13 & 8192 & 805306368 &  \SI{25.8}{\arcsecond}\\
\multicolumn{1}{c|}{\vdots} & \multicolumn{1}{c|}{\vdots} & \multicolumn{1}{c|}{\vdots} & \multicolumn{1}{c}{\vdots} \\
\end{tabular}
\caption[HEALPix parameters and resulting angular resolutions]{\textbf{HEALPix parameters and resulting angular resolutions.} $k$ represents the number of dividing iterations on the 12 panes, $N_\text{side}$ the number of tiles per pane edge, $N_\text{pix}$ the total number of pixels, and $\theta_\text{pix}$ the angular resolution defined by the angular length of a pixel edge. \cite{healpix:paper}}
\end{table}

For the application in this simulation the pixelization of a whole sphere is not needed since the telescope only has a field of view of about \SI{12}{\degree} (i.e. $\theta \leq \SI{6}{\degree}$). Considering that, one only needs a smaller sector of pixels around the zenith at $\theta = \SI{0}{\degree}$ which reduces the number of needed HEALPix by a factor of
\begin{align}
	\Gamma = \frac{1}{4\pi}\int\limits_{0}^{2\pi}\int\limits_{0}^{\theta_\text{max}}\sin{\theta} d\theta d\phi = \frac{1-\cos\theta_\text{max}}{2}\,.
\end{align}
For the IceAct simulation with $\theta_\text{max} = \SI{10}{\degree}$ this leads to the fact that only about \SI{0.76}{\percent} of the HEALPixes are needed.

\section{Lookup Table}

\section{CORSIKA Implementation}\todo{vermutlich quatsch}

    \chapter{Application on Simulated Data}

The parameterization strategy introduced in chapter~\ref{chap:param_strategy} can now be applied to the simulation results from \geant. In the following sections, some comparisons and results are shown and discussed.

\section{Wavelength Binning}\label{sec:wvl_binning}

Since the photon detection efficiency of the SiPMs is a non-constant function of the wavelength (cf. figure~\ref{sipm:pde}), one can optimize the different wavelength ranges or \textit{bin sizes}~$\Delta\lambda$ by equalizing not the bin sizes (\textit{constant binning}) but the detected photons per bin which is further referred to as \textit{adaptive binning}. Figure~\ref{param:wvl_binning} shows a comparison between constant and adaptive wavelength binning.

\begin{figure}[H]
	\centering
	\begin{subfigure}[t]{0.49\textwidth}
		\includegraphics[width=\textwidth]{constant_wvl_bins_hist.pdf}
		\subcaption{constant binning}
		\label{param:wvl_binning:constant}
	\end{subfigure}
	\hfill
	\begin{subfigure}[t]{0.49\textwidth}
		\includegraphics[width=\textwidth]{lut_wvl_bins_hist.pdf}
		\subcaption{adaptive binning}
		\label{param:wvl_binning:adaptive}
	\end{subfigure}
	\caption[Constant vs. adaptive wavelength binning]{\textbf{Constant vs. adaptive wavelength binning.} All wavelengths of photons that are detected by any camera SiPM are histogramized in \num{50} bins between \SI{272}{\nano\meter} and \SI{900}{\nano\meter}. In (\subref{param:wvl_binning:constant}), the \num{50} bins are distributed uniformly in wavelength so that one can see a shape that is quite similar to the PDE of the SiPMs. Figure~(\subref{param:wvl_binning:adaptive}) shows the same data with adaptive bin sizes to equalize the counts per bin. The resulting bin edges are rounded to integer numbers which causes the visible fluctuations. Since the adaptive bin limits are always included by closed intervals, there are slightly more detected photons in~(\subref{param:wvl_binning:adaptive}) than in~(\subref{param:wvl_binning:constant}) due to numerical issues. This is not a problem since also the simulated photons are binned in the same closed intervals which result in correct detection efficiency calculations.}
	\label{param:wvl_binning}
\end{figure}

The adaptive bin edges are calculated by sorting the wavelengths of all simulated particles that are detected by any of the 61 SiPM, i.e. all photons on which the KDE will be applied afterwards (cf. section~\ref{sec:adaptive_vs_nonadaptive}). Next, this sequence is divided into \num{50} parts of equal length. Thus, the wavelengths are divided into consecutive \SI{2}{\percent}-quantiles. In order to get more convenient bin sizes, the quantile limits (or bin edges) are rounded to integer values which obviously result in some fluctuations (cf. figure~\ref{param:wvl_binning:adaptive}).

With the adaptive wavelength binning it is ensured that in each range $\Delta\lambda$ almost the same number of photons is detected which enables a statistically more stable probability density estimation. The simulated wavelength range starts at $\lambda_\text{min}=\SI{272}{\nano\meter}$ since there is no photon detected with a wavelength below \SI{272}{\nano\meter} due to the absorption properties or the glass plate (cf. \ref{sec:iceact:model:material}).

\section{Adaptive vs. Non-adaptive KDE}\label{sec:adaptive_vs_nonadaptive}

Now that the wavelength binning is defined, the next step is to take a look on the camera pixels individually. Due to the distinct field of view of each pixel, the direction distribution of each pixel is characteristic and has regions with very different statistical densities as already stated in section~\ref{sec:adaptive_kde}. In order to get an idea of the given direction distributions for which the KDE should be calculated, figure \ref{param:example_scatter} shows some exemplary scatter plots of detected photons by an arbitrary camera pixel~$i$ in a wavelength range~$\Delta\lambda$.\\

\begin{figure}[H]
	\centering
	\begin{subfigure}[t]{0.49\textwidth}
		\includegraphics[width=\textwidth]{scatter_px00_wvl450-455nm.png}
		\subcaption{}
		\label{param:example_scatter:1}
	\end{subfigure}
	\hfill
	\begin{subfigure}[t]{0.49\textwidth}
		\includegraphics[width=\textwidth]{scatter_px10_wvl800-900nm.png}
		\subcaption{}
		\label{param:example_scatter:2}
	\end{subfigure}
	\vfill
	\begin{subfigure}[b]{0.49\textwidth}
		\includegraphics[width=\textwidth]{scatter_px20_wvl410-415nm.png}
		\subcaption{}
		\label{param:example_scatter:3}
	\end{subfigure}
	\hfill
	\begin{subfigure}[b]{0.49\textwidth}
		\includegraphics[width=\textwidth]{scatter_px54_wvl600-615nm.png}
		\subcaption{}
		\label{param:example_scatter:4}
	\end{subfigure}
	\caption[Example: directions of detected photons as a scatter plot]{\textbf{Example: directions of detected photons as a scatter plot.} A subset of simulated photon directions that are detected in a given camera pixel $i$ and the wavelength range $\Delta\lambda$ are shown in a polar plot. One can clearly see that there are regions with high and low statistical significance. Additionally, the \textit{ghost image} effect (cf. section~\ref{sec:ghost_image}) is visible in the non-central pixels ((\subref{param:example_scatter:2}), (\subref{param:example_scatter:3}), (\subref{param:example_scatter:4})).}
	\label{param:example_scatter}	
\end{figure}

The regions with very sparse \enquote{dots} can only arise from random scattering processes inside the optical system since they are outside the main field of view and the \textit{ghost image} region (cf. section~\ref{sec:ghost_image}). Thus, the probability density should be rather constant in these scattering regions. For the KDE, one achieves this by increasing the kernel bandwith. Simultaneously, the \enquote{real} detection regions should be described precisely which is done by reducing the bandwith. The need of an adaptive kernel density approach is given. Figure~\ref{param:kde_comparison} strikingly shows the difference between an adaptive and a non-adaptive KDE. Since the KDEs calculated there are only based on a small subsample of the simulation data, they may give the impression that the adaptive KDE still is very inaccurate in the scattering regions but on the full data sample this is not the case any more by having more scattered photons.\\

\begin{figure}[H]
	\centering
	\begin{subfigure}[t]{0.40\textwidth}
		\includegraphics[width=\textwidth]{comparison_adaptive_nonadaptive_px00_wvl450-455.pdf}
		\subcaption{}
		\label{param:kde_comparison:1}
	\end{subfigure}
	\hfill
	\begin{subfigure}[t]{0.40\textwidth}
		\includegraphics[width=\textwidth]{comparison_adaptive_nonadaptive_px10_wvl800-900.pdf}
		\subcaption{}
		\label{param:kde_comparison:2}
	\end{subfigure}
	\vfill
	\begin{subfigure}[b]{0.40\textwidth}
		\includegraphics[width=\textwidth]{comparison_adaptive_nonadaptive_px20_wvl410-415.pdf}
		\subcaption{}
		\label{param:kde_comparison:3}
	\end{subfigure}
	\hfill
	\begin{subfigure}[b]{0.40\textwidth}
		\includegraphics[width=\textwidth]{comparison_adaptive_nonadaptive_px54_wvl600-615.pdf}
		\subcaption{}
		\label{param:kde_comparison:4}
	\end{subfigure}
	\caption[Comparison: adaptive vs. non-adaptive KDE]{\textbf{Comparison: adaptive vs. non-adaptive KDE.} Evaluation of the direction distributions shown in figure~\ref{param:example_scatter} on a HEALPix grid with $N_\text{side}=\num{512}$. The plot shows a disc up to $\theta=\SI{10}{\degree}$ and the white dotted meridians have an azimuth distance of $\SI{30}{\degree}$. The azimuth $\phi$ starts at the solid white line and goes around counter-clockwise. Differences between an adaptive (top) and non-adaptive (bottom) KDE approach are visible. Especially in the region with low statistics (scattering region), the non-adaptive KDE is dominated by the fluctuations while adaptive KDE blurs the probability density more strongly.}
	\label{param:kde_comparison}		
\end{figure}

Anyway, the detection distributions can now be calculated and renormalized for each of the \num{61} camera pixels and \num{50} wavelength bins which result in \num{3050} so called \textit{detection efficiency maps} shown in the next section~\ref{sec:deteff_maps}.

\section{Detection Efficiency Maps}\label{sec:deteff_maps}

By following the strategy described in chapter~\ref{chap:param:strategy} and considering the findings from sections~\ref{sec:wvl_binning} and \ref{sec:adaptive_vs_nonadaptive}, one can now finally calculate \textit{detection efficiency maps}. For each camera pixel $i$ and wavelength bin $\Delta\lambda$, these maps show the probability $\epsilon_{i,\Delta\lambda,HP}$ to detect a photon with wavelength $\lambda\in\Delta\lambda$ in camera pixel $i$ as a function of its origin direction~$(\theta,\phi)$ which is coded in the ordinal number~$HP$ of the corresponding HEALPix. For the error estimation, the bootstrapping method presented in section~\ref{sec:bootstrapping} is used. To get a confident estimation with reasonable computational effort, $N_\text{bootstrap} = \num{10}$ bootstrapping iterations are performed.\\

\subsection{Choice of HEALPix Pixelization Parameter $N_\text{side}$}

For the pixelization parameter of the HEALPix model~$N_\text{side}$, one has to choose a reasonable value as well (cf. table~\ref{healpix:table}): too fine pixelization would obviously result in a very detailed parameterization. Due to KDE, an \enquote{unbinned} detection efficiency is available -- at least in origin directions -- so that it is technically possible to choose a very fine binning. The problem is, that the main goal of the \iceact parameterization is to produce a lookup table that is efficient and capable of evaluating large amounts of Cherenkov photons. An unnecessarily fine HEALPix binning would just blow up the lookup table and quick evaluation is not feasible anymore. To determine the optimal $N_\text{side}$ of the HEALPix model, the detection efficiency of the central pixel $i=0$ is considered. As one can see in figure~\ref{deteffmap:px0}, the most efficient angular area is below a zenith angle of $\theta=\SI{0.7}{\degree}$. Usually, this regions are called \textit{field of view} (\textit{FOV}) which equals the doubled zenith angle $2\theta$ if the FOV is symmetrical around the zenith. Thus, the core field of view of the central pixel is $\SI{1.4}{\degree}$. In the transition region $\SI{1.4}{\degree} < 2\theta \leq \SI{2.5}{\degree}$, the detection efficiency decreases rapidly. Therefore, is it crucial to pixelize this region such that the hexagonal shape of the pixel's field of view -- which is caused by the Winston cones -- is described in a sufficient degree of detail in order to have a fairly precise approximation.

\begin{figure}[H]
	\centering
	\includegraphics[width=\textwidth]{deteffmaps/deteff_wvl413nm_px00_view1.5deg.pdf}
	\caption[Detection efficiency of central camera pixel]{\textbf{Detection efficiency of central camera pixel.} The plot is zoomed into the main detection region up to a zenith angle $\theta=\SI{1.5}{\degree}$. As a wavelength, the bin around the most efficient wavelength~$\lambda^\ast$ (cf. section~\ref{sec:best_wvl}) is chosen. In a field of view below $\SI{1.4}{\degree}$, the central pixel has its maximal detection efficiency. The HEALPix parameter in this map is $N_\text{side}=\num{512}$. Bootstrapping leads to the relative error shown in the right plot. The interval between the white dashed meridians is $\Delta\phi=\SI{10}{\degree}$.}
	\label{deteffmap:px0}
\end{figure}

As figure~\ref{deteffmap:px0} shows, a HEALPix parameter of $N_\text{side}=\num{512}$ is sufficient to describe the hexagonal shape. By applying $N_\text{side}=\num{256}$, four HEALPix would reduce to just one HEALPix, which could hardly describe the edges. On the other side, one gets to $N_\text{side}=\num{1024}$ by dividing a HEALPix into four. This results in a unnecessarily high amount of HEALPix to describe the given shape. Table~\ref{n_healpix_fov} shows the number of enclosed HEALPix(-centers) in the core region ($2\theta\leq\SI{1.4}{\degree}$) and the transition region ($\SI{1.4}{\degree} < 2\theta \leq \SI{2.5}{\degree}$) depending on the used HEALPix parameter.

\begin{table}[H]
	\centering
	\begin{tabular}{c|c|c}
		\toprule
		\multirow{2}{*}{$N_\text{side}$} & \multicolumn{2}{c}{HEALPixes in region} \\
		&	$2\theta\leq\SI{1.4}{\degree}$ & $\SI{1.4}{\degree} < 2\theta \leq \SI{2.5}{\degree}$ \\
		\midrule
		\num{256}  & \num{24}  & \num{88} \\
		\num{512}  & \num{112} & \num{368} \\
		\num{1024} & \num{480} & \num{1380} \\
		\bottomrule
	\end{tabular}
	\caption[Number of HEALPixes centers in regions around the zenith]{\textbf{Number of HEALPixes centers in regions around the zenith.} The numbers for three possible HEALPix models are shown. As characteristic areas, the efficiency core region ($2\theta\leq\SI{1.4}{\degree}$) and transition region ($\SI{1.4}{\degree} < 2\theta \leq \SI{2.5}{\degree}$) of the central pixel around the most efficient wavelength~$\lambda^\ast$ (cf. section~\ref{sec:best_wvl}) are chosen.}
	\label{n_healpix_fov}
\end{table}

\subsection{Overview of Results}

In this section, some results are presented. At first, the focus is on the maximum detection efficiency. Just because the optics has been optimized to the \enquote{best} wavelength $\lambda^\ast=\SI{411}{\nano\meter}$ by minimizing the focal spot for this wavelength (cf. section~\ref{sec:focalplaneshift}), this does not necessarily imply that the detection efficiency is maximal at $\lambda^\ast$. Indeed, the maximal efficiency is reached between \SI{426}{\nano\meter} and \SI{431}{\nano\meter} for the central camera pixel with $\epsilon_\text{max}\approx\SI{33.35}{\percent}$ as figure~\ref{max_deteff} shows. One also observes, that for the off-axis pixels -- i.e. all but the central one -- the centroid of the efficiency distributions slightly shifts towards higher wavelengths which. The reason is that the focal plane shifts \enquote{behind} the camera plane with increasing wavelengths (cf. figure~\ref{focalplaneshift}). The higher the wavelength, the more blurred the image gets and these wavelengths are tendentially seen by more pixels. As a result, more inner pixels are more sensitive to lower and more outer pixels are more sensitive to higher wavelengths.\\

\begin{figure}[H]
	\centering
	\includegraphics[width=0.7\textwidth]{deteffmaps/deteff_bins_max.pdf}
	\caption[Maximum detection efficiency for each pixel and wavelength]{\textbf{Maximum detection efficiency for each pixel and wavelength.} The 61 camera pixels are color-coded via their ordinal number from blue to read. The overall maximum detection efficiency is reached by the central pixel at $\Delta\lambda=[\SI{426}{\nano\meter},\SI{431}{\nano\meter}]$ with $\epsilon_\text{max}\approx\SI{33.35}{\percent}$.}
	\label{max_deteff}
\end{figure}

By summing up the individual detection efficiency maps of each pixel, the response of the whole camera to a certain wavelength is given. Figure~\ref{deteff:full_cam} shows these maps for three characteristic wavelength ranges. What is remarkable is that in the region of focus, i.e. wavelengths for which the focal spot on the camera plane is relatively small, the geometry of the camera is important. One can see very sharp differences between the main field of view of each pixel which corresponds to the Winston cone centers and the Winston cone edges resulting in an approximately \SI{10}{\percent} lower efficiency (cf. figure~\ref{deteff:full_cam:2}). For smaller wavelengths where the focus is above the camera plane, a distinct geometry-dependent pattern is not visible except, of course, the overall hexagonal shape. For higher wavelengths, a pattern is again visible, but a barrel-shaped distortion is visible. This is a known optical aberration effect where the magnification decreases with distance from the optical axis like for an image of a fisheye lens.\\

\begin{figure}[H]
	\centering
	\begin{subfigure}[t]{0.705\textwidth}
		\includegraphics[width=\textwidth]{deteffmaps/deteff_wvl296nm_pxall_view10deg.pdf}
		\subcaption{Ultra-violet range.}
		\label{deteff:full_cam:1}
	\end{subfigure}
	\hfill
	\begin{subfigure}[t]{0.705\textwidth}
		\includegraphics[width=\textwidth]{deteffmaps/deteff_wvl428.5nm_pxall_view10deg.pdf}
		\subcaption{Maximal efficient range.}
		\label{deteff:full_cam:2}
	\end{subfigure}
	\vfill
	\begin{subfigure}[t]{0.705\textwidth}
		\includegraphics[width=\textwidth]{deteffmaps/deteff_wvl782nm_pxall_view10deg.pdf}
		\subcaption{High wavelengths range.}
		\label{deteff:full_cam:3}
	\end{subfigure}
	\caption[Detection efficiency maps of the full camera]{\textbf{Detection efficiency maps of the full camera.} The left plots show the detection efficiency while the right plots show the relative error. Three wavelength regions are chosen: a UV range in (\subref{deteff:full_cam:1}), the maximum efficiency in (\subref{deteff:full_cam:2}) and high wavelengths in (\subref{deteff:full_cam:3}). The white dashed parallels and meridians have the distances $\Delta\theta=\SI{2}{\degree}$ and $\Delta\phi=\SI{10}{\degree}$. The azimuth revolves counter-clockwisely starting at the white solid line. Please note the different color mapping for the detection efficiencies.}
	\label{deteff:full_cam}		
\end{figure}

Interesting optical effects are visible if one plots some off-axis pixels individually. Figure~\ref{deteff:offaxis_px} shows three of these. Besides the core region one can also see the ghost image explained in section~\ref{sec:ghost_image} on the opposide side in terms of the azimuth. Additionally, there are some spots next to the core and the ghost image region which originate from reflections between the Winston cones' top and Fresnel lens' back side. In blind regions where the detection probability does not exceed \SI{e-3}{\percent}, the estimated efficiency is highly dominated by statistical fluctuations as the plots for relative statistical errors show as well.\\

\begin{figure}[H]
	\centering
	\begin{subfigure}[t]{0.705\textwidth}
		\includegraphics[width=\textwidth]{deteffmaps/deteff_wvl428.5nm_px11_view10.0deg.pdf}
		\subcaption{Pixel 11 on symmetry axis along $y$.}
		\label{deteff:offaxis_px:1}
	\end{subfigure}
	\hfill
	\begin{subfigure}[t]{0.705\textwidth}
		\includegraphics[width=\textwidth]{deteffmaps/deteff_wvl428.5nm_px14_view10.0deg.pdf}
		\subcaption{Pixel 14 on symmetry axis along $x$.}
		\label{deteff:offaxis_px:2}
	\end{subfigure}
	\vfill
	\begin{subfigure}[t]{0.705\textwidth}
		\includegraphics[width=\textwidth]{deteffmaps/deteff_wvl428.5nm_px36_view10.0deg.pdf}
		\subcaption{Pixel 36 on \enquote{quasi-symmetry} axis.}
		\label{deteff:offaxis_px:3}
	\end{subfigure}
	\caption[Detection efficiency maps for some off-axis pixels]{\textbf{Detection efficiency for maps some off-axis pixels.} Similar plots as in figure~\ref{deteff:full_cam} but now for individual camera pixels at the wavelength range with maximum efficiency. Three pixels on symmetry axes are chosen. The two pixels shown in (\subref{deteff:offaxis_px:1}) and (\subref{deteff:offaxis_px:2}) are on real symmetry axes while the third pixel in (\subref{deteff:offaxis_px:3}) is on a symmetry axis of the camera plane but not with respect to the squared SiPMs.}
	\label{deteff:offaxis_px}		
\end{figure}

\section{Lookup Table (LUT)}

As the detection efficiency maps are ready now, one has to think about a way to access the information as fast as possible. Usually, this is done by storing the parameterization function evaluated at certain points in a multidimensional array structure known as \textit{lookup table}. Afterwards, events can be \enquote{diced} with the given information yielding count histograms for the camera -- named \textit{images}.

\subsection{LUT Production}\label{sec:lut_production}

First of all, one has to define how to iterate over the given information. In this case, one starts with a photon direction $(\theta,\phi)$ translated into the corresponding HEALPix number~$HP$ and the wavelength~$\lambda$ of this photon which is assigned to the proper wavelength bin~$\Delta\lambda$. The result then should be the response of each pixel to the very same photon, i.e. how probable is it for the photon to be detected in camera pixel $i$. In order to achieve this, one can evaluate the detection efficiency maps by fist considering just one HEALPix. For this HEALPix, the detection efficiency for all wavelength bins and all camera pixels is read out, which is shown exemplary in figure~\ref{lut:transpose_example}. 

\begin{figure}[H]
	\includegraphics[width=\textwidth]{LUT_transposition_example.pdf}
	\caption[Visualization for the lookup table transposition]{\textbf{Visualization for the lookup table transposition.} A certain HEALPix (here: $HP=\num{4890}$) is considered. For one camera pixel, the detection efficiency is read from the maps throughout all wavelength bins. In this example, some maps for pixel \num{20} are chosen. Doing this for all camera pixels yields to the blue-framed plot. For the considered HEALPix one gets the wavelength-dependent detection efficiency for each camera pixel. The \enquote{camera top view} plot shows the camera as seen from the lens. Pixels that have a maximum detection efficiency greater than~\SI{0.1}{\percent} are emphasized with colors. The detection efficiency functions of all other pixels are not plotted. Besides, the HEALPix center coordinates are given with a polar plot.}
	\label{lut:transpose_example}
\end{figure}

By iterating over each HEALPix of one gets $N_{HP}\times N_{\Delta\lambda}$ pixel-by-pixel detection efficiency arrays where $N_{HP}$ is the total number of HEALPix covered by simulation and $N_{\Delta\lambda}$ the number of wavelength bins. These sub-arrays are further named as $\epsilon_{HP,\Delta\lambda}$ and contain the detection efficiency $\epsilon_i$ for each camera pixel $i$ (also called \textit{pixel response}). 

Additionally, one can even improve the structure of these sub-array by not just saving the single responses $\epsilon_i$ ordered by pixel number, but saving the cumulative responses. Then, the $k$-th of totally \num{60} elements in the sub-array is defined as 
\begin{align}
	\epsilon^\text{abs}_k = \sum_{i=0}^{k} \epsilon_i\,,
	\label{eq:eps_cumulative}
\end{align}
i.e. $\epsilon^\text{abs}_k$ is the probability to detect the photon in any pixel with number $i\in[0,k]$. Additionally, it is
\begin{align}
\epsilon_k^\text{abs} < 1\,,
\end{align}
where the last element $\epsilon_{60}^\text{abs}$ is the total detection probability of the camera and distances between the elements represent the responses of the individual camera pixels. The advantage of saving the cumulative sum rather than actual responses gets clear in the next section~\ref{sec:lut_readout}. Schematically, the lookup process can be described by
\begin{align}
	\gamma \rightarrow
	\begin{cases}
		(\theta,\phi) & \rightarrow HP\\
		\lambda & \rightarrow \Delta\lambda
	\end{cases}
	\Rightarrow \epsilon^\text{abs}_{HP,\Delta\lambda} = \{\epsilon^\text{abs}_k\}_{k=0}^{60}\,.
\end{align}\\

The HEALPix resolution used for parameterization of \iceact is $N_\text{side}=\num{512}$. With $\theta_\text{max}=\SI{10}{\degree}$, this results in a total \enquote{active} HEALPix number of $N_{HP}=\num{24420}$. Additionally, the wavelengths are divided into $N_{\Delta\lambda} = \num{50}$ bins in the interval $[\SI{272}{\nano\meter},\SI{900}{\nano\meter}]$ (cf. section~\ref{sec:wvl_binning}). For the \num{61} pixel camera this results in $N_{HP}\cdot N_{\Delta\lambda}\cdot \num{61} = \num{74481000}$ numbers to be saved. The needed disk space for this lookup table is \SI{284}{\mebi\byte} by using \SI{32}{\bit} float as data type\footnote{Since also the information about wavelength binning, HEALPix model, and maximum simulated zenith angle has to be included, the lookup table file might need slightly more space}.

\subsection{LUT Readout -- \enquote{Event Dicing}}\label{sec:lut_readout}

Now that the data is ready, an evaluation algorithm has to be elaborated. In the following, the principle to evaluate $N$ photons is described step by step.

\begin{enumerate}
	\item For each direction the corresponding HEALPix number is calculated.\footnote{In this thesis, the evaluation algorithm is implemented with Python. For HEALPix calculations, the package \textit{healpy} is used.}
	\begin{align}
		\{(\theta_i, \phi_i)\}_{i=1}^N \rightarrow \{HP_i\}_{i=1}^N
	\end{align}
	\item Only valid photons are evaluated. A photon is valid if its HEALPix number~$HP_i$ is in the lookup table -- i.e. below the cutoff -- and if its wavelength~$\lambda_i$ is in the wavelength range covered by the lookup table $[\lambda_\text{min},\lambda_\text{max}]$. $N^\ast\leq N$ photons are valid. Invalid photons are counted.
	\begin{align}
		\text{valid} \coloneqq HP_i < N_{HP} \wedge \lambda_i \in [\lambda_\text{min},\lambda_\text{max}]
	\end{align} 
	\item For all $N^\ast$ photons, the corresponding wavelength bin is determined.
	\begin{align}
		\{\lambda_{i^\ast}\}_{i^\ast=1}^{N^\ast} \rightarrow \{ \Delta\lambda_{i^\ast} \}_{i^\ast=1}^{N^\ast}
	\end{align}
	Here, the index of the assigned wavelength bin rather than the actual range is important for the lookup table query.
	\item The actual lookup table query is performed. Thus, every valid photon gets its proper pixel response array.
	\begin{align}
		\{HP_{i^\ast}, \Delta\lambda_{i^\ast}\}_{i^\ast=1}^{N^\ast} \rightarrow \{ \epsilon^\text{abs}_{HP_{i^\ast},\Delta\lambda_{i^\ast}}\}_{i^\ast=1}^{N^\ast}
	\end{align}
	\item For each valid photon $i^\ast$, a random number $r_{i^\ast}\in[0,1)$ is diced and sorted into the corresponding response array. Now, the advantage of saving the cumulative rather than the absolute responses (cf. section~\ref{sec:lut_production}) becomes clear. By definition, the cumulative responses are sorted in ascending order\footnote{This is the case if all numbers are non-negative like it is the case for the camera pixel responses. Otherwise, the cumulative sequence may not be sorted.}. As a result, the position $k$ where the random number is sorted in equals the ordinal number of the camera pixel where the photon is detected. If the random number is sorted in at the \num{61}-th position -- i.e. at the end -- this means that the photon is not detected. The cumulative saving enables a lookup process that is only based on comparative operations which are very fast.
	\item The resulting sorting positions $k_{i^\ast}$ can now be histogramized which gives the final image seen by the camera.
\end{enumerate}

\section{Application on CORSIKA Air Shower Events}

With the lookup table, one can finally produce event displays -- or images -- of simulated air showers. A commonly used toolkit for detailed simulation of cosmic-ray air showers is \textit{CORSIKA}\footnote{\textbf{CO}smic \textbf{R}ay \textbf{S}imulations for \textbf{KA}scade}~\cite{corsika:website}. It is capable of simulating the trajectory and energy of Cherenkov photons which emerge from air showers.

\todo{...}
\todo{Plots ...}
    \chapter{Summary and Outlook}

\todo{now possible: evaluation of Monte Carlo data}

\todo{adaptable for HAWCs Eye, FAMOUS, ...}

\todo{Signal simulation}

% -----------------------------------
\appendix
	% !TeX spellcheck = en_US
\chapter{Access to the IceAct \geant Simulation}

The \iceact \geant simulation is available in the RWTH Aachen GitLab:\\

\url{https://git.rwth-aachen.de/iceact/simulation}.\\
 
To get access, please contact
\begin{itemize}
	\item \href{mailto:maurice.guender@rwth-aachen.de}{maurice.guender@rwth-aachen.de}, or
	\item \href{mailto:erik.ganster@rwth-aachen.de}{erik.ganster@rwth-aachen.de}. 
\end{itemize}

Besides the \geant simulation, the repository contains Python scripts for evaluation.

\chapter{Additional Plots}

\section{Impact-location-dependent Detection Efficiency}\label{appendix:alienplots}

\begin{figure}[H]
	\centering
	\begin{subfigure}[t]{0.492\textwidth}
		\fbox{\includegraphics[width=\textwidth]{alienplots/theta1_phi090.png}}
		\subcaption{}
	\end{subfigure}
	\hfill
	\begin{subfigure}[t]{0.492\textwidth}
		\fbox{\includegraphics[width=\textwidth]{alienplots/theta2_phi090.png}}
		\subcaption{}
	\end{subfigure}
	\hfill
\end{figure}

\begin{figure}[H]
	\ContinuedFloat
	\centering
	\begin{subfigure}[t]{0.492\textwidth}
		\fbox{\includegraphics[width=\textwidth]{alienplots/theta3_phi090.png}}
		\subcaption{}
	\end{subfigure}
	\hfill
	\begin{subfigure}[t]{0.492\textwidth}
		\fbox{\includegraphics[width=\textwidth]{alienplots/theta5_phi090.png}}
		\subcaption{}
	\end{subfigure}
	\hfill
	\begin{subfigure}[t]{0.492\textwidth}
		\fbox{\includegraphics[width=\textwidth]{alienplots/theta6_phi090.png}}
		\subcaption{}
	\end{subfigure}	
	\hfill
	\begin{subfigure}[t]{0.492\textwidth}
		\fbox{\includegraphics[width=\textwidth]{alienplots/theta7_phi090.png}}
		\subcaption{}
	\end{subfigure}
\end{figure}

\section{More Detection Efficiency Maps}

\section{Camera Pixel Response For Some HEALPix}

\begin{figure}[H]
	\centering
	\begin{subfigure}[t]{0.495\textwidth}
		\includegraphics[width=\textwidth]{deteffmaps/deteff_healpix00000.pdf}
		\subcaption{}
	\end{subfigure}
	\hfill
	\begin{subfigure}[t]{0.495\textwidth}
		\includegraphics[width=\textwidth]{deteffmaps/deteff_healpix00500.pdf}
		\subcaption{}
	\end{subfigure}
	\hfill
	\begin{subfigure}[t]{0.495\textwidth}
		\includegraphics[width=\textwidth]{deteffmaps/deteff_healpix01000.pdf}
		\subcaption{}
	\end{subfigure}
	\hfill
	\begin{subfigure}[t]{0.495\textwidth}
		\includegraphics[width=\textwidth]{deteffmaps/deteff_healpix01500.pdf}
		\subcaption{}
	\end{subfigure}
	\hfill
	\begin{subfigure}[t]{0.495\textwidth}
		\includegraphics[width=\textwidth]{deteffmaps/deteff_healpix02000.pdf}
		\subcaption{}
	\end{subfigure}
	\hfill
	\begin{subfigure}[t]{0.495\textwidth}
		\includegraphics[width=\textwidth]{deteffmaps/deteff_healpix02500.pdf}
		\subcaption{}
	\end{subfigure}
	\hfill
	\begin{subfigure}[t]{0.495\textwidth}
		\includegraphics[width=\textwidth]{deteffmaps/deteff_healpix03000.pdf}
		\subcaption{}
	\end{subfigure}
	\hfill
	\begin{subfigure}[t]{0.495\textwidth}
		\includegraphics[width=\textwidth]{deteffmaps/deteff_healpix03500.pdf}
		\subcaption{}
	\end{subfigure}
\end{figure}

\begin{figure}[H]
	\ContinuedFloat
	\centering
	\begin{subfigure}[t]{0.495\textwidth}
		\includegraphics[width=\textwidth]{deteffmaps/deteff_healpix04000.pdf}
		\subcaption{}
	\end{subfigure}
	\hfill
	\begin{subfigure}[t]{0.495\textwidth}
		\includegraphics[width=\textwidth]{deteffmaps/deteff_healpix04500.pdf}
		\subcaption{}
	\end{subfigure}
	\hfill
	\begin{subfigure}[t]{0.495\textwidth}
		\includegraphics[width=\textwidth]{deteffmaps/deteff_healpix05000.pdf}
		\subcaption{}
	\end{subfigure}
	\hfill
	\begin{subfigure}[t]{0.495\textwidth}
		\includegraphics[width=\textwidth]{deteffmaps/deteff_healpix05500.pdf}
		\subcaption{}
	\end{subfigure}
	\hfill
	\begin{subfigure}[t]{0.495\textwidth}
		\includegraphics[width=\textwidth]{deteffmaps/deteff_healpix06000.pdf}
		\subcaption{}
	\end{subfigure}
	\hfill
	\begin{subfigure}[t]{0.495\textwidth}
		\includegraphics[width=\textwidth]{deteffmaps/deteff_healpix06500.pdf}
		\subcaption{}
	\end{subfigure}
	\hfill
	\begin{subfigure}[t]{0.495\textwidth}
		\includegraphics[width=\textwidth]{deteffmaps/deteff_healpix07000.pdf}
		\subcaption{}
	\end{subfigure}
	\hfill
	\begin{subfigure}[t]{0.495\textwidth}
		\includegraphics[width=\textwidth]{deteffmaps/deteff_healpix07500.pdf}
		\subcaption{}
	\end{subfigure}
	\hfill
	\begin{subfigure}[t]{0.495\textwidth}
		\includegraphics[width=\textwidth]{deteffmaps/deteff_healpix08000.pdf}
		\subcaption{}
	\end{subfigure}
	\hfill
	\begin{subfigure}[t]{0.495\textwidth}
		\includegraphics[width=\textwidth]{deteffmaps/deteff_healpix08500.pdf}
		\subcaption{}
	\end{subfigure}
\end{figure}

\begin{figure}
	\ContinuedFloat
	\centering
	\begin{subfigure}[t]{0.495\textwidth}
		\includegraphics[width=\textwidth]{deteffmaps/deteff_healpix09000.pdf}
		\subcaption{}
	\end{subfigure}
	\hfill
	\begin{subfigure}[t]{0.495\textwidth}
		\includegraphics[width=\textwidth]{deteffmaps/deteff_healpix09500.pdf}
		\subcaption{}
	\end{subfigure}
	\hfill
	\begin{subfigure}[t]{0.495\textwidth}
		\includegraphics[width=\textwidth]{deteffmaps/deteff_healpix10000.pdf}
		\subcaption{}
	\end{subfigure}
	\hfill
	\begin{subfigure}[t]{0.495\textwidth}
		\includegraphics[width=\textwidth]{deteffmaps/deteff_healpix10500.pdf}
		\subcaption{}
	\end{subfigure}
	\hfill
	\begin{subfigure}[t]{0.495\textwidth}
		\includegraphics[width=\textwidth]{deteffmaps/deteff_healpix11000.pdf}
		\subcaption{}
	\end{subfigure}
	\hfill
	\begin{subfigure}[t]{0.495\textwidth}
		\includegraphics[width=\textwidth]{deteffmaps/deteff_healpix11500.pdf}
		\subcaption{}
	\end{subfigure}
	\hfill
	\begin{subfigure}[t]{0.495\textwidth}
		\includegraphics[width=\textwidth]{deteffmaps/deteff_healpix12000.pdf}
		\subcaption{}
	\end{subfigure}
	\hfill
	\begin{subfigure}[t]{0.495\textwidth}
		\includegraphics[width=\textwidth]{deteffmaps/deteff_healpix12500.pdf}
		\subcaption{}
	\end{subfigure}
\end{figure}
\backmatter
	\pagenumbering{Roman} 
	\renewcommand{\listfigurename}{Figures}
	\listoffigures 				% Abbildungsverzeichnis
	\renewcommand{\listtablename}{Tables}
	\listoftables				% Tabellenverzeichnis
	\printbibliography[title=Literature]
	% !TeX spellcheck = de_DE
\chapter{Acknowledgements\\Danksagungen}

Nach nunmehr einem Jahr gilt es, einigen Personen zu danken.\\

2016 habe ich meine Bachelorarbeit ebenfalls am III. Physikalischen Institut B bei Prof. Wiebusch angefertigt, weshalb mir viele Gesichter der Aachener \icecube Arbeitsgruppe schon bekannt waren. Die Atmosphäre innerhalb der Gruppe ist sehr harmonisch und der rege Austausch in Meetings und Telefonkonferenzen sowie häufiges Feedback haben maßgeblich zu einem Gelingen dieser Ausarbeitung beigetragen.\\

Insbesondere möchte ich mich bei Prof. Wiebusch bedanken, der mir die Gelegenheit gegeben hat, meine Masterarbeit in seiner Gruppe anzufertigen und Teil einer internationalen Kollaboration zu sein.\\

Besonderer Dank gilt auch meinem Betreuer Jan Auffenberg und meinen Bürokollegen Merlin Schaufel und Erik Ganster, die mir durch häufige und inspirierende Gespräche jederzeit mit Rat und Tat zur Seite standen.\\

Abschließend bedanke ich mich bei meiner Familie und meinen Freunden, die mich während meines gesamten Studiums unterstützt und mir Rückhalt gegeben haben.

\includepdf{./chapters/form_eidesstattliche_versicherung.pdf}

\end{document}