% !TeX spellcheck = en_US
\documentclass[12pt,twoside,paper=a4,bibliography=totoc, listof=totoc,headsepline,footsepline,chapterprefix=true,open=right,headings=twolinechapter,headings=big]{scrbook}
% -----------------------------------
% für große Kapitelzahlen
\usepackage{anyfontsize}
%\setkomafont{chapter}{\normalfont\Huge}
\renewcommand*{\chapterheadstartvskip}{\vspace*{4\baselineskip}}
\renewcommand*{\chapterheadendvskip}{\vspace*{2\baselineskip}}
\renewcommand*{\chapterformat}{%
	{\fontsize{20}{30}\scshape\chapappifchapterprefix{}}%
	\fontsize{90}{30}\selectfont\rlap{\thechapter\autodot}%
}
\renewcommand*{\raggedchapter}{\raggedright}

\renewcommand*\othersectionlevelsformat[3]{%
	\llap{#3\autodot\enskip}%
}
\usepackage[backend=biber,style=numeric-comp,sorting=none,backref=true,sortcites=true]{biblatex}
\addbibresource{bibliography.bib}	
\renewcommand*{\bibfont}{\small}
\usepackage{ragged2e}
\AtBeginBibliography{\RaggedRight}
\makeatletter
\def\blx@maxline{77}
% für URLs in Quellen und im Text
\makeatother	
\usepackage{url}		
\usepackage{mdwlist}
% Sprache und Codierung
\usepackage[utf8]{inputenc}
%\usepackage[ngerman]{babel}
\usepackage[T1]{fontenc}
\usepackage{csquotes}
% Kopf- und Fußzeilen
\usepackage{scrlayer-scrpage}
\ofoot*{\raisebox{-2mm}{\pagemark}}
\renewcommand*{\chapterpagestyle}{scrheadings}
% Fußnoten nicht auf mehrere Seiten aufteilen
\interfootnotelinepenalty=10000
% Absatzanfang nicht einrücken
\setlength{\parindent}{0pt}
% Grafiken
\usepackage{graphicx}
\usepackage{pdfpages}
\graphicspath{{./images/}{./plots/}{./sketches/}}
\usepackage[space]{grffile}
% Tabellen
\usepackage{tabularx}
\usepackage{longtable}
\usepackage{multirow}
\usepackage{listings}
% Bildunterschriften
\usepackage{float}
\usepackage[format=plain, indention=1.5em, labelfont={bf}, justification=justified]{caption}
% Bilder neben Text
\usepackage{wrapfig}
% Zeileneinstellung
\usepackage{setspace} 
% Mathe		
\usepackage{amsmath}
\usepackage{amsfonts}
\usepackage{amssymb}
\usepackage{mathtools}
% Counter
\usepackage{chngcntr}
\counterwithout{footnote}{chapter}
% Gibt an, wie viele Ebenen nummeriert werden sollen
\setcounter{secnumdepth}{5}
\setcounter{tocdepth}{5}
% Einheiten
\usepackage[per-mode=reciprocal, separate-uncertainty=true]{siunitx}
\sisetup{mode=text,range-phrase = {\text{~to~}}}
% Serifenschrift
\rmfamily
% Seitenränder
\usepackage[outer=28mm, inner=18mm, top=32mm, bottom=36mm]{geometry}
% Hyperlinks
\usepackage{hyperref}
% Todonotes
\usepackage{todonotes}
% Schriftart
%\usepackage{lmodern}
\renewcommand{\familydefault}{\sfdefault}
% für Subcaptions
\usepackage{subcaption}
% mehere Referenzen in einem
\usepackage{cleveref}
% Umgebung für Variablenbeschreibung
\newenvironment{vardescription}
{\par\vspace{\abovedisplayskip}\noindent
	\tabularx{\columnwidth}{>{$}l<{$} @{${}\quad{}$} >{\raggedright\arraybackslash}X}}
{\endtabularx\par\vspace{\belowdisplayskip}}
\usepackage{xspace}
% Um zwei Subfigures aneinander zentriert auszurichten
\newsavebox{\savedimage}
\newcommand{\saveimageheight}[2][]{%
	\savebox{\savedimage}{\includegraphics[#1]{#2}}}
\newcommand{\raiseimage}[2][]{%
	\raisebox{.5\dimexpr\ht\savedimage-\height}{%
		\includegraphics[#1]{#2}}}%
% Zeilenumbruch nach "paragraph"
\makeatletter
\renewcommand\paragraph{\@startsection{paragraph}{4}{\z@}%
	{-3.25ex\@plus -1ex \@minus -.2ex}%
	{1.5ex \@plus .2ex}%
	{\normalfont\normalsize\bfseries}}
\makeatother
% Geant4
\newcommand{\geant}{\textsc{Geant4}\xspace}

% Document
% -----------------------------------
\begin{document}

\frontmatter 
	% !TeX spellcheck = en_US
\begin{titlepage}
\addtolength{\oddsidemargin}{4mm}
\begin{center}

% Titel der Arbeit 
\LARGE
\textbf{Simulation of the Optics of the Imaging Air Cherenkov Telescope IceAct with Geant4} \\[15mm]
% Simulation der Optik des abbildenden Luft Tscherenkow Teleskops IceAct mit Geant4

% Autor
{\large by}\\[1mm]
\Large
Maurice Günder, B.Sc.\\[18mm]

% Art der Arbeit
Master's Thesis in Physics\\[70mm]

{\large in}\\[1mm]
April 2019\\[35mm]

Univ.-Prof. Dr. Christopher Wiebusch \\\textsc{III. Physikalisches Institut B \\Rheinisch- Westfälische Technische Hochschule Aachen}

\end{center}
\end{titlepage} 		% Titelblatt
    % !TeX spellcheck = en_US
\cleardoublepage

\vspace*{15cm}

\begin{flushleft}
	\textbf{First Referee}\\
	Univ.-Prof. Dr. Christopher Wiebusch\\
	Physics Institute III B\\
	RWTH Aachen University\\
	\bigskip
	
	\textbf{Second Referee}\\
	Prof. Dr. Thomas Bretz\\
	Physics Institute III A\\
	RWTH Aachen University
	\bigskip
	
	\textbf{Supervisor}\\
	Dr. Jan Auffenberg\\
	Physics Institute III B\\
	RWTH Aachen University\\
\end{flushleft}		% Betreuer, Korrektor
    \onehalfspacing             % Zeilenabstand ab hier 1.5 zeilig
	\renewcommand{\contentsname}{Content}
	\tableofcontents			% Inhaltsverzeichnis
	\protect\thispagestyle{scrheadings}

% -----------------------------------
\mainmatter 					% die einzelnen Kapitel
    \pagestyle{scrheadings}
    % !TeX spellcheck = en_US
\chapter{Introduction}

\section{Motivation}

\section{The IceCube Neutrino Observatory}

Since January 2011, the IceCube Neutrino Observatory at the South Pole is measuring neutrinos emanating from various sources. For this purpose a detector instrumented with digital optical modules (\enquote{DOMs}) is installed deep in the antarctic ice. 5160 of these optical sensors are arranged on 86 strings at a height between \SI{1450}{\meter} and \SI{2450}{\meter} below the surface. The central region of this In-Ice Array which has a higher density of DOMs is called \enquote{DeepCore}.

\begin{figure}[h]
	\includegraphics[width=\textwidth]{IceCubeDetector.pdf}
	\caption[Schematic view of IceCube]{\textbf{Schematic view of the IceCube Neutrino Observatory. \cite{icecube:instrumentation}} The in-ice array with the denser sub-array DeepCore as well as the surface array IceTop is sketched. Different station colors represent different deployment stages.}
\end{figure}

Neutrinos are very interesting elementary particles because of their weak interaction cross section and their electrical neutrality. This fact makes it possible for neutrinos to let them point back to their sources which is exploited in the search for astrophysical processes like active galactic nuclei, supernovae, or gamma-ray bursts. Since they are able to reach us without scattering processes, neutrinos can even give information about sources at cosmological distances. Simultaneously, the weak interaction potential is what neutrino detection makes challenging. Therefore, a detector with a large scale active volume is needed. In the case of IceCube, this is \SI{1}{\cubic\kilo\meter} of ice.

At the surface on top of the in-ice detector the cosmic ray air shower array IceTop is installed to detect Cherenkov radiation (see \ref{sec:cherenkov}). IceTop consists of 81 stations approximately arranged in the same grid as the in-ice strings. Each station has two tanks filled up with ice and two standard IceCube DOMs. This arrangement makes it possible for IceTop to detect primary cosmic rays (see \ref{sec:cosmicrays}) in the energy range of \si{\peta\electronvolt} to \si{\exa\electronvolt}. \cite{icecube:instrumentation}

\section{Air Showers and Cosmic Rays}\label{sec:cosmicrays}

\section{Atmospheric Cherenkov Light}\label{sec:cherenkov}

\section{Imaging Air Cherenkov Telescopes}

\subsection{Imaging Technique}

\subsection{Photon Detection}

\subsection{IceAct}
    % !TeX spellcheck = en_US
\chapter{The IceAct Model in \geant}

\section{\geant}
\geant is a multi-purpose simulation framework for the passage of particles trough matter, written in \textit{C++} and developed by the \geant Collaboration at CERN. It includes physics models, geometry, tracking, hits, and digitization and thus allows a detailed simulation and response analysis for particle detectors in many application fields like particle and accelerator physics, space engineering or medical science. In the framework's source some basic and advanced use cases are implemented and provided as examples. The toolkit is built up of multiple categories (or modules) using each other (cf. figure \ref{geant4:categories}). \cite{geant4}

\begin{wrapfigure}{r}{0.5\textwidth}
	\centering
	\includegraphics[width=0.5\textwidth]{Geant4Logo.png}
	\caption[\geant Logo]{\textbf{\geant logo.} \cite{geant4:logo}}	
\end{wrapfigure}

\begin{figure}[h]
	\centering
	\includegraphics[width=0.6\textwidth]{Geant4ConceptDiagram.pdf}
	\caption[\geant category diagram]{\textbf{Diagram of relationships between \geant categories. \cite{geant4}} The circles represent a \enquote{using} relation. The category with the circle next to the box uses the linked one.}	
	\label{geant4:categories}
\end{figure}

Especially for the use case of IceAct \geant is capable of simulating Cherenkov (optical) photons, material properties like transmission, reflection, and refraction, as well as detection efficiency properties of the SiPMs.

Since this thesis is about an approach of an all-encompassing telescope simulation, \geant provides all major possibilities to get a distinct analysis of the entire optical system of IceAct.

\subsection{FAMOUS telescope simulation}
The fluorescence telescope FAMOUS\footnote{\textbf{F}irst \textbf{A}uger \textbf{M}ulti-pixel
photon counter camera for the \textbf{O}bservation of \textbf{U}ltra-high-energy air \textbf{S}howers} for the Pierre Auger Observatory in Argentina is developed at RWTH Aachen to measure fluorescence light originating from ultra-high-energy cosmic rays (UHECR) by using Silicon Photomultipliers (SiPMs). Withing the development, a detailed \geant simulation has been elaborated. \cite{famous:sim_github,famous:sim_github} The telescope design of FAMOUS is very similar since the detection technique and the optics system is basically the same. Therefore, the IceAct telescope simulation is heavily based on this FAMOUS \geant framework. A detailed discourse and a summary of previous analyses can be found in \cite{famous:niggemann}.

\section{Materials}

For an optical device, the material that the light should pass has to be chosen deliberately. Especially the transmission properties, processability, and for IceAct in particular the resistance against harsh weather conditions are of interest.

The glass plate on top of IceAct is made of SCHOTT BOROFLOAT\textsuperscript{\textregistered} 33 borosilicate glass with a thickness of \SI{2+-0.2}{\milli\meter} and a diameter of \SI{650.3+-1}{\milli\meter}. Borosilicate is chosen for its high durability, transparency in the interesting spectral region, flatness, and weak fluorescence intensities. The refractive index is evaluated at some wavelength. Since we need to have a full dispersion relation the points are spline interpolated (cf. orange curve in figure \ref{iceact:model:material:refractive_index}). \cite{iceact:borosilicate:datasheet}. 

In the data sheet \cite{iceact:borosilicate:datasheet} the transmission properties are given for a vertical light and a glass plate of a thickness of $d = \SI{6.5}{\milli\meter}$. Therefore, the transmission curve $T_\text{total}(\lambda)$ includes the internal absorption as well as the two interface transitions into and out of the borosilicate.
\begin{align}
	T_\text{total}(\lambda) = T_\text{interface}^2(\lambda)\cdot T_\text{internal}(d=\SI{6.5}{\milli\meter},\lambda)
	\label{eq:transmission}
\end{align}
The transmission at the interface can be calculated by using the Fresnel equations. In case of perpendicular light, it is
\begin{align}
	T_\text{interface}(\lambda) = 1 - \left(\frac{n(\lambda)-n_\text{air}}{n(\lambda)+n_\text{air}}\right)^2
	\label{eq:perp_interface_transmission}
\end{align}
In \geant the wavelength dependent absorption length $a(\lambda)$ has to be implemented which is given by the exponential absorption law
\begin{align}
	I(x) = I_0 e^{-\frac{x}{a}} \Leftrightarrow a = - \frac{x}{\ln\left(\frac{I(x)}{I_0}\right)}
	\label{eq:absorptionlaw}
\end{align}
Thus, one gets the absorption length by using equations \eqref{eq:transmission}, \eqref{eq:perp_interface_transmission}, and \eqref{eq:absorptionlaw}.
\begin{align}
	a(\lambda) &= - \frac{d}{\ln T_\text{internal}(d,\lambda)}\\
	&= -\frac{d}{\ln T_\text{total}(\lambda)-2\ln\left(1 - \left(\frac{n(\lambda)-n_\text{air}}{n(\lambda)+n_\text{air}}\right)^2\right)}\,,
\end{align}
which is implemented in \geant material properties with $n_\text{air} = 1$. Figure \ref{iceact:model:material:transmission} shows the three transmission components as orange lines.

The Fresnel lens and the Winston Cones in IceAct are made of polymethyl methacrylate (PMMA, acrylic, or plexiglass). The dispersion $n(\lambda)$ can be parametrized with the empirical \textit{Sellmeier equation}. For glasses the usual form is
\begin{align}
	n^2(\lambda) = 1 + \frac{B_1\lambda^2}{\lambda^2-C_1} + \frac{B_2\lambda^2}{\lambda^2-C_2} + \frac{B_3\lambda^2}{\lambda^2-C_3}\,,
	\label{eq:sellmeier}
\end{align}
with the \textit{Sellmeier coefficients} $B_{1,2,3}$ and $C_{1,2,3}$. \cite{iceact:sellmeier} Table \ref{iceact:model:pmma_sellmeiercoeffs} shows the used coefficients and the function is plotted in figure \ref{iceact:model:material:refractive_index} (blue curve).

\begin{table}[h]
	\centering
	\begin{tabular}{c|l}
		$B_1$  & \num{0.99654}  \\
		$B_2$  & \num{0.18964}  \\
		$B_3$  & \num{0.00411}  \\
		$C_1$  & \SI{0.00787}{\micro\meter\squared}  \\
		$C_2$  & \SI{0.02191}{\micro\meter\squared}  \\
		$C_3$  & \SI{3.85727}{\micro\meter\squared}  \\
	\end{tabular}
	\caption[Sellmeier coefficients for PMMA]{\textbf{Sellmeier coefficients for PMMA.} \cite{iceact:refractiveindex} The above-mentioned coefficients are used in the \geant material properties for PMMA. The related Sellmeier equation \eqref{eq:sellmeier} is plotted in figure \ref{iceact:model:material:refractive_index} as the blue curve.}
	\label{iceact:model:pmma_sellmeiercoeffs}
\end{table}

For the transmission properties of PMMA, the same method as for borosilicate is used (see above). Therefore, the data stated in \cite{famous:niggemann} is taken as $T_\text{internal}(d = \SI{3}{\milli\meter})$. Figure \ref{iceact:model:material:transmission} shows the three transmission components as blue lines.

\begin{figure}[h]
	\centering
	\includegraphics[width=0.7\textwidth]{material/transmission.pdf}
	\caption[Transmission of used materials]{\textbf{Transmission functions of materials used in the simulation.} The total transmission function is the product of internal and two interface transmissions which is evaluated for a perpendicularly incident particle in this plot. Thus, the solid lines represent a complete (perpendicular) transition through a \SI{3}{\milli\meter} thick layer of the respective material. For a better comparison, the data of internal transmission for borosilicate is converted from \SI{6.5}{\milli\meter} into \SI{3}{\milli\meter}.}
	\label{iceact:model:material:transmission}	
\end{figure}

\begin{figure}[h]
	\centering
	\includegraphics[width=0.7\textwidth]{material/refractive_index.pdf}
	\caption[Refractive index of used materials]{\textbf{Refractive index of materials used in the simulation.} For PMMA the dispersion is calculated by evaluating the Sellmeier equation introduced in this section. The refractive index for the used borosilicate is only given at specific wavelengths. \cite{iceact:borosilicate:datasheet} Therefore, spline interpolation is used to reconstruct the full curve.}
	\label{iceact:model:material:refractive_index}	
\end{figure}

The tube, back plane and other coating surfaces are simulated as \enquote{dummy} material with no reflection or transmission parameters. A particle that hits those surfaces is absorbed and not considered any further.

\section{Optics}

In the following section the four optical components of the IceAct \geant model are discussed. To have a first glimpse of the model, see figure \ref{iceact:model:cut}.
\begin{figure}[h]
	\centering
	\includegraphics[width=0.6\textwidth]{IceActGeant4Model.pdf}
	\caption[IceAct \geant model]{\textbf{The IceAct \geant model.} Cross-sectional sketch of the IceAct optics in \geant with all simulated components. They are described in detail in \cref{iceact:model:fresnellens,iceact:model:camera,iceact:model:glassplate}.}
	\label{iceact:model:cut}	
\end{figure}

\subsection{Glass Plate}\label{iceact:model:glassplate}

\todo{glass plate}

\subsection{Fresnel Lens}\label{iceact:model:fresnellens}

\begin{figure}[h]
	\centering
	\includegraphics[width=0.5\textwidth]{FresnelVsNormalLens.pdf}
	\caption[Comparison conventional vs. Fresnel lens]{\textbf{Comparison between a conventional \enquote{thick} lens and a Fresnel lens. \cite{famous:eichler}} For the functionality of a lens the radius-dependent sagitta function $z(\rho)$ is crucial. To get rid of the bulky material of a conventional lens, the Fresnel lens is divided into annular \enquote{prisms} called \enquote{grooves}. The slope angle $\delta$ of each groove is a local approximation of the sagitta function to ensure the imaging capability.}
	\label{iceact:model:fresnelvsthick}	
\end{figure}

\begin{figure}[h]
	\centering
	\includegraphics[width=0.5\textwidth]{FresnelGroove.pdf}
	\caption[Fresnel groove]{\textbf{Cross-sectional sketch of a Fresnel groove. \cite{famous:eichler}}}
	\label{iceact:model:fresnelgroove}	
\end{figure}

\todo{write}
\cite{famous:eichler}

\subsection{Camera}\label{iceact:model:camera}

\todo{write}

\subsubsection{Winston Cones}

\todo{write}

\subsubsection{Silicon Photomultiplier}

\todo{write}


    % !TeX spellcheck = en_US
\chapter{Simulation Results}

\section{Simulation Strategy and Verification}

\subsection{Winston Cone Meshing}

\subsection{Aberration Effects -- \enquote{Ghost Image}}

\section{Simulation of Single Components}

\subsection{\enquote{Best} Wavelength}

\subsection{Focal Plane Shift}
    % !TeX spellcheck = en_US
\chapter{IceAct Parameterization}

The simulation results discussed in chapter \ref{chap:simresults} can now be used to parameterize the telescope response of IceAct. The major goal of this is to provide a fast way to evaluate the detection probability of incident photons in each camera pixel which is done by elaboration of a lookup table (\textit{LUT}). In the following sections, the method of setting up this LUT is described.

\section{Kernel Density Estimation}

Kernel density estimation \textit{KDE} is a non-parametric method to estimate a probability density function of a random variable by a given finite data sample. The standard \textit{kernel density estimator}
\begin{align}
	\hat{f}(x)=\frac{1}{nh}\sum_{i=1}^{n}K\left(\frac{x-X_i}{h}\right)\,,
	\label{eq:kde}
\end{align}
is the sum of \textit{kernel functions} $K(\dots)$ for each data point $X_i$. The non-negative parameter $h$ is the \textit{bandwidth} and is a measure for the smoothing of the resulting KDE: the KDE gets smoother with increasing $h$. Due to the normalization factor, $\hat{f}(x)$ is normed to
\begin{align}
	\int\limits_{-\infty}^{+\infty}\hat{f}(x)dx \equiv 1\,.
\end{align}

\begin{figure}[h]
	\centering
	\includegraphics[width=0.5\textwidth]{KDE_example.pdf}
	\caption[Example KDE with different smoothing]{\textbf{Example KDE with different smoothing.} \cite{kde:example_plot} Kernel density estimation is applied on a data sample with 100 random numbers drawn from a normal distribution (gray curve). The blue, green, and orange curves have different bandwidths.}
	\label{kde:example_1d}	
\end{figure}

The kernel function can be a very distinct function that can in principle describe any probability density. In this parameterization method a Gaussian kernel
\begin{align}
	K(t) = \frac{1}{\sqrt{2\pi}}e^{-\frac{1}{2}t^2}
\end{align}
is used.

In order to get the KDE describing the probability density appropriately, one has to choose a reasonable bandwith for the given data sample size. As shown in figure \ref{kde:example_1d}, too low bandwith results in a spiky, fluctuating KDE. If the bandwidth is too high, one might get a rather inaccurate estimator. Therefore it seems reasonable to choose different bandwidth in regions with different amount of statistics which then is called \textit{adaptive} kernel density estimation. \cite{kde:schoenen, kde:wangwang}

\subsection{Adaptive KDE with Gaussian Kernel}
As one can see in the simulation results (cf. \todo{verweis zu map}) and in figure \ref{kde:example_scatter}, each pixel has a certain region of photon directions where it is efficient. However, each pixel is almost \enquote{blind} for most other directions. This results in statistically stable -- \enquote{dense} -- regions but also \enquote{sparse} regions dominated by scattered photons that undergo large statistical fluctuations. Thus, and based on the former conclusions, an approach to adapt the bandwidth to the local statistics should perform well. In this thesis, an algorithm presented by \textsc{B. Wang} and \textsc{X. Wang} in \cite{kde:wangwang} which has been implemented within the scope of \cite{kde:schoenen} is used. The adaptive (in principle weighted) kernel density estimator is calculated by \cite{kde:schoenen,kde:wangwang}
\begin{align}
	f(\vec{x}) = \sum_{i=0}^{n} \frac{w_i}{N_i}e^{-\frac{1}{2}(\vec{x}-\vec{X_i})^T \frac{1}{h\lambda_i} \mathbf{C}^{-1} (\vec{x}-\vec{X_i})}\,,
\end{align}
with
\begin{vardescription}
	n & total number of data points,\\
	w_i & weight of th $i$-th data point,\\
	N_i & Gaussian normalization,\\
	\vec{X_i} & coordinate vector of the $i$-the data point,\\
	h & global bandwidth factor,\\
	\lambda_i & local bandwidth factor,\\
	\mathbf{C} & covariance matrix.\\
\end{vardescription}
The global bandwidth factor $h$ is calculated by the \textit{Silverman rule} \cite{kde:schoenen,kde:wangwang}
\begin{align}
	h = \left(\frac{n(d+2)}{4}\right)^{-\frac{1}{d+4}}\,,
\end{align}
where $d$ equals the number of dimensions of the data (here $d=2$). The local bandwidth $\lambda_i$ is the factor where the local statistics of each data point comes in. It is defined as \cite{kde:schoenen,kde:wangwang}
\begin{align}
	\lambda_i = \left(\frac{\hat{f}(\vec{X_i})}{g}\right)^{-\alpha}\,,
\end{align}
with
\begin{vardescription}
	\hat{f}(\vec{X_i}) = f(\vec{X_i})|_{w_i=\lambda_i=1}\,,\\
	\ln{g} = n^{-1} \sum_{i=0}^{n} \ln{\hat{f}(\vec{X_i})}\,,\\
	\alpha\in[0,1]\,.
\end{vardescription}
For the IceAct parameterization the sensitivity parameter $\alpha$ is set to $\alpha=\num{0.3}$, since this has been shown to be an appropriate value as well as in \cite{kde:schoenen}. Additionally, all $n$ photon hits have the same weight, which results in \cite{kde:schoenen,kde:wangwang}
\begin{align}
	w_i = \frac{1}{n}\,.
\end{align}
To get an idea of the given data points for which the KDE should be calculated, figure \ref{kde:example_scatter} shows a scatter plot of detected photons for an arbitrary camera pixel in a possible range of wavelengths. The need for an adaptive KDE approach is clearly visible by regions with very different statistical densities. In addition, figure \ref{kde:comparison} shows strikingly the difference between an adaptive and a non-adaptive KDE.

\begin{figure}[h]
	\centering
	\includegraphics[width=0.5\textwidth]{scatter_px20_wvl410-420nm.png}
	\caption[Example: directions of detected photons as a scatter plot]{\textbf{Example: directions of detected photons as a scatter plot.} A subset of simulated photon directions that are detected in camera pixel 20 with wavelengths between \SI{410}{\nano\meter} and \SI{420}{\nano\meter} are shown in a polar plot. One can clearly see that there are regions with high and low statistical significance.}
	\label{kde:example_scatter}	
\end{figure}

\begin{figure}[h]
	\centering
	\includegraphics[width=\textwidth]{comparison_adaptive_nonadaptive.pdf}
	\caption[Comparison: adaptive vs. non-adaptive KDE]{\textbf{Comparison: adaptive vs. non-adaptive KDE.} The difference between an adaptive (left) and non-adaptive (right) KDE approach is visible. Especially in the region with low statistics, the non-adaptive KDE is dominated by the fluctuations.}
	\label{kde:comparison}	
\end{figure}

\subsection{Bootstrapping}

One problem of kernel density estimation is that it does not provide any statistical information, i.e. \enquote{how precise} the probability density is estimated.

\section{Detection Efficiency}

\subsection{The HEALPix Algorithm}

In order to account for the spherical shape of the angles of incidence it is useful to have angular bins with equal areas. Therefore, in this parameterization method, the \textit{HEALPix} (\textbf{H}ierarchical \textbf{E}qual \textbf{A}rea iso\textbf{L}atitude \textbf{Pix}elization) algorithm is used. It allows a uniform pixelization of incidence angles to the telescope.\\

\todo{healpix pixelization image}

The basic idea is to subdivide the sphere into 12 quadrilateral panes which can then further divided into more sub-panes as shown in figure \todo{figure reference}. A parameter $k$ numbers the subdivision step so that the number of sub-panes per each of the 12 panes is $N_{\text{side}}^2=\left(2^k\right)^2$. Hence, the total amount of pixels on the sphere is then
\begin{align}
N_\text{pix} = 12N_\text{side}^2\,.
\end{align}
Obviously, the angular resolution is dependent on $N_\text{pix}$, henceforth referred to as \enquote{pixel}.

\begin{table}[h]
\centering
\begin{tabular}{S[table-format=2.0]|S[table-format=4.0]|S[table-format=9.0]|S[table-format=2.2]}
\multicolumn{1}{l|}{$k$} & \multicolumn{1}{l|}{$N_\text{side} = 2^k$} & \multicolumn{1}{l|}{$N_\text{pix} = 12N_\text{nside}^2$} & \multicolumn{1}{l}{$\theta_\text{pix}$} \\
\hline
0  & 1    & 12        &  \SI{58.6}{\degree}\\
1  & 2    & 48        &  \SI{29.3}{\degree}\\
2  & 4    & 192       &  \SI{14.7}{\degree}\\
3  & 8    & 768 	  &  \SI{7.33}{\degree}\\
4  & 16   & 3072      &  \SI{3.66}{\degree}\\
5  & 32   & 12288     &  \SI{1.83}{\degree}\\
6  & 64   & 49152     &  \SI{55.0}{\arcminute}\\
7  & 128  & 196608    &  \SI{27.5}{\arcminute}\\
8  & 256  & 786432    &  \SI{13.7}{\arcminute}\\
9  & 512  & 3145728   &  \SI{6.87}{\arcminute}\\
10 & 1024 & 12582912  &  \SI{3.44}{\arcminute}\\
11 & 2048 & 50331648  &  \SI{1.72}{\arcminute}\\
12 & 4096 & 201326592 &  \SI{51.5}{\arcsecond}\\
13 & 8192 & 805306368 &  \SI{25.8}{\arcsecond}\\
\multicolumn{1}{c|}{\vdots} & \multicolumn{1}{c|}{\vdots} & \multicolumn{1}{c|}{\vdots} & \multicolumn{1}{c}{\vdots} \\
\end{tabular}
\caption[HEALPix parameters and resulting angular resolutions]{\textbf{HEALPix parameters and resulting angular resolutions.} $k$ represents the number of dividing iterations on the 12 panes, $N_\text{side}$ the number of tiles per pane edge, $N_\text{pix}$ the total number of pixels, and $\theta_\text{pix}$ the angular resolution defined by the angular length of a pixel edge. \cite{healpix:paper}}
\end{table}

For the application in this simulation the pixelization of a whole sphere is not needed since the telescope only has a field of view of about \SI{12}{\degree} (i.e. $\theta \leq \SI{6}{\degree}$). Considering that, one only needs a smaller sector of pixels around the zenith at $\theta = \SI{0}{\degree}$ which reduces the number of needed HEALPix by a factor of
\begin{align}
	\Gamma = \frac{1}{4\pi}\int\limits_{0}^{2\pi}\int\limits_{0}^{\theta_\text{max}}\sin{\theta} d\theta d\phi = \frac{1-\cos\theta_\text{max}}{2}\,.
\end{align}
For the IceAct simulation with $\theta_\text{max} = \SI{10}{\degree}$ this leads to the fact that only about \SI{0.76}{\percent} of the HEALPixes are needed.

\section{Lookup Table}

\section{CORSIKA Implementation}\todo{vermutlich quatsch}


% -----------------------------------
\backmatter
	\renewcommand{\listfigurename}{Figures}
	\listoffigures 				% Abbildungsverzeichnis
	\protect\thispagestyle{scrheadings}
	\renewcommand{\listtablename}{Tables}
	\listoftables				% Tabellenverzeichnis
	\protect\thispagestyle{scrheadings}
	\pagestyle{scrheadings}
	\printbibliography[title=Literature]

\end{document}